\documentclass[11pt]{article}

    \usepackage[breakable]{tcolorbox}
    \usepackage{parskip} % Stop auto-indenting (to mimic markdown behaviour)
    
    \usepackage{iftex}
    \ifPDFTeX
    	\usepackage[T1]{fontenc}
    	\usepackage{mathpazo}
    \else
    	\usepackage{fontspec}
    \fi

    % Basic figure setup, for now with no caption control since it's done
    % automatically by Pandoc (which extracts ![](path) syntax from Markdown).
    \usepackage{graphicx}
    % Maintain compatibility with old templates. Remove in nbconvert 6.0
    \let\Oldincludegraphics\includegraphics
    % Ensure that by default, figures have no caption (until we provide a
    % proper Figure object with a Caption API and a way to capture that
    % in the conversion process - todo).
    \usepackage{caption}
    \DeclareCaptionFormat{nocaption}{}
    \captionsetup{format=nocaption,aboveskip=0pt,belowskip=0pt}

    \usepackage{float}
    \floatplacement{figure}{H} % forces figures to be placed at the correct location
    \usepackage{xcolor} % Allow colors to be defined
    \usepackage{enumerate} % Needed for markdown enumerations to work
    \usepackage{geometry} % Used to adjust the document margins
    \usepackage{amsmath} % Equations
    \usepackage{amssymb} % Equations
    \usepackage{textcomp} % defines textquotesingle
    % Hack from http://tex.stackexchange.com/a/47451/13684:
    \AtBeginDocument{%
        \def\PYZsq{\textquotesingle}% Upright quotes in Pygmentized code
    }
    \usepackage{upquote} % Upright quotes for verbatim code
    \usepackage{eurosym} % defines \euro
    \usepackage[mathletters]{ucs} % Extended unicode (utf-8) support
    \usepackage{fancyvrb} % verbatim replacement that allows latex
    \usepackage{grffile} % extends the file name processing of package graphics 
                         % to support a larger range
    \makeatletter % fix for old versions of grffile with XeLaTeX
    \@ifpackagelater{grffile}{2019/11/01}
    {
      % Do nothing on new versions
    }
    {
      \def\Gread@@xetex#1{%
        \IfFileExists{"\Gin@base".bb}%
        {\Gread@eps{\Gin@base.bb}}%
        {\Gread@@xetex@aux#1}%
      }
    }
    \makeatother
    \usepackage[Export]{adjustbox} % Used to constrain images to a maximum size
    \adjustboxset{max size={0.9\linewidth}{0.9\paperheight}}

    % The hyperref package gives us a pdf with properly built
    % internal navigation ('pdf bookmarks' for the table of contents,
    % internal cross-reference links, web links for URLs, etc.)
    \usepackage{hyperref}
    % The default LaTeX title has an obnoxious amount of whitespace. By default,
    % titling removes some of it. It also provides customization options.
    \usepackage{titling}
    \usepackage{longtable} % longtable support required by pandoc >1.10
    \usepackage{booktabs}  % table support for pandoc > 1.12.2
    \usepackage[inline]{enumitem} % IRkernel/repr support (it uses the enumerate* environment)
    \usepackage[normalem]{ulem} % ulem is needed to support strikethroughs (\sout)
                                % normalem makes italics be italics, not underlines
    \usepackage{mathrsfs}
    

    
    % Colors for the hyperref package
    \definecolor{urlcolor}{rgb}{0,.145,.698}
    \definecolor{linkcolor}{rgb}{.71,0.21,0.01}
    \definecolor{citecolor}{rgb}{.12,.54,.11}

    % ANSI colors
    \definecolor{ansi-black}{HTML}{3E424D}
    \definecolor{ansi-black-intense}{HTML}{282C36}
    \definecolor{ansi-red}{HTML}{E75C58}
    \definecolor{ansi-red-intense}{HTML}{B22B31}
    \definecolor{ansi-green}{HTML}{00A250}
    \definecolor{ansi-green-intense}{HTML}{007427}
    \definecolor{ansi-yellow}{HTML}{DDB62B}
    \definecolor{ansi-yellow-intense}{HTML}{B27D12}
    \definecolor{ansi-blue}{HTML}{208FFB}
    \definecolor{ansi-blue-intense}{HTML}{0065CA}
    \definecolor{ansi-magenta}{HTML}{D160C4}
    \definecolor{ansi-magenta-intense}{HTML}{A03196}
    \definecolor{ansi-cyan}{HTML}{60C6C8}
    \definecolor{ansi-cyan-intense}{HTML}{258F8F}
    \definecolor{ansi-white}{HTML}{C5C1B4}
    \definecolor{ansi-white-intense}{HTML}{A1A6B2}
    \definecolor{ansi-default-inverse-fg}{HTML}{FFFFFF}
    \definecolor{ansi-default-inverse-bg}{HTML}{000000}

    % common color for the border for error outputs.
    \definecolor{outerrorbackground}{HTML}{FFDFDF}

    % commands and environments needed by pandoc snippets
    % extracted from the output of `pandoc -s`
    \providecommand{\tightlist}{%
      \setlength{\itemsep}{0pt}\setlength{\parskip}{0pt}}
    \DefineVerbatimEnvironment{Highlighting}{Verbatim}{commandchars=\\\{\}}
    % Add ',fontsize=\small' for more characters per line
    \newenvironment{Shaded}{}{}
    \newcommand{\KeywordTok}[1]{\textcolor[rgb]{0.00,0.44,0.13}{\textbf{{#1}}}}
    \newcommand{\DataTypeTok}[1]{\textcolor[rgb]{0.56,0.13,0.00}{{#1}}}
    \newcommand{\DecValTok}[1]{\textcolor[rgb]{0.25,0.63,0.44}{{#1}}}
    \newcommand{\BaseNTok}[1]{\textcolor[rgb]{0.25,0.63,0.44}{{#1}}}
    \newcommand{\FloatTok}[1]{\textcolor[rgb]{0.25,0.63,0.44}{{#1}}}
    \newcommand{\CharTok}[1]{\textcolor[rgb]{0.25,0.44,0.63}{{#1}}}
    \newcommand{\StringTok}[1]{\textcolor[rgb]{0.25,0.44,0.63}{{#1}}}
    \newcommand{\CommentTok}[1]{\textcolor[rgb]{0.38,0.63,0.69}{\textit{{#1}}}}
    \newcommand{\OtherTok}[1]{\textcolor[rgb]{0.00,0.44,0.13}{{#1}}}
    \newcommand{\AlertTok}[1]{\textcolor[rgb]{1.00,0.00,0.00}{\textbf{{#1}}}}
    \newcommand{\FunctionTok}[1]{\textcolor[rgb]{0.02,0.16,0.49}{{#1}}}
    \newcommand{\RegionMarkerTok}[1]{{#1}}
    \newcommand{\ErrorTok}[1]{\textcolor[rgb]{1.00,0.00,0.00}{\textbf{{#1}}}}
    \newcommand{\NormalTok}[1]{{#1}}
    
    % Additional commands for more recent versions of Pandoc
    \newcommand{\ConstantTok}[1]{\textcolor[rgb]{0.53,0.00,0.00}{{#1}}}
    \newcommand{\SpecialCharTok}[1]{\textcolor[rgb]{0.25,0.44,0.63}{{#1}}}
    \newcommand{\VerbatimStringTok}[1]{\textcolor[rgb]{0.25,0.44,0.63}{{#1}}}
    \newcommand{\SpecialStringTok}[1]{\textcolor[rgb]{0.73,0.40,0.53}{{#1}}}
    \newcommand{\ImportTok}[1]{{#1}}
    \newcommand{\DocumentationTok}[1]{\textcolor[rgb]{0.73,0.13,0.13}{\textit{{#1}}}}
    \newcommand{\AnnotationTok}[1]{\textcolor[rgb]{0.38,0.63,0.69}{\textbf{\textit{{#1}}}}}
    \newcommand{\CommentVarTok}[1]{\textcolor[rgb]{0.38,0.63,0.69}{\textbf{\textit{{#1}}}}}
    \newcommand{\VariableTok}[1]{\textcolor[rgb]{0.10,0.09,0.49}{{#1}}}
    \newcommand{\ControlFlowTok}[1]{\textcolor[rgb]{0.00,0.44,0.13}{\textbf{{#1}}}}
    \newcommand{\OperatorTok}[1]{\textcolor[rgb]{0.40,0.40,0.40}{{#1}}}
    \newcommand{\BuiltInTok}[1]{{#1}}
    \newcommand{\ExtensionTok}[1]{{#1}}
    \newcommand{\PreprocessorTok}[1]{\textcolor[rgb]{0.74,0.48,0.00}{{#1}}}
    \newcommand{\AttributeTok}[1]{\textcolor[rgb]{0.49,0.56,0.16}{{#1}}}
    \newcommand{\InformationTok}[1]{\textcolor[rgb]{0.38,0.63,0.69}{\textbf{\textit{{#1}}}}}
    \newcommand{\WarningTok}[1]{\textcolor[rgb]{0.38,0.63,0.69}{\textbf{\textit{{#1}}}}}
    
    
    % Define a nice break command that doesn't care if a line doesn't already
    % exist.
    \def\br{\hspace*{\fill} \\* }
    % Math Jax compatibility definitions
    \def\gt{>}
    \def\lt{<}
    \let\Oldtex\TeX
    \let\Oldlatex\LaTeX
    \renewcommand{\TeX}{\textrm{\Oldtex}}
    \renewcommand{\LaTeX}{\textrm{\Oldlatex}}
    % Document parameters
    % Document title
    \title{Qiskit\_Notes-Chapter\_1-UF\_211129 }
    
    
    
    
    
% Pygments definitions
\makeatletter
\def\PY@reset{\let\PY@it=\relax \let\PY@bf=\relax%
    \let\PY@ul=\relax \let\PY@tc=\relax%
    \let\PY@bc=\relax \let\PY@ff=\relax}
\def\PY@tok#1{\csname PY@tok@#1\endcsname}
\def\PY@toks#1+{\ifx\relax#1\empty\else%
    \PY@tok{#1}\expandafter\PY@toks\fi}
\def\PY@do#1{\PY@bc{\PY@tc{\PY@ul{%
    \PY@it{\PY@bf{\PY@ff{#1}}}}}}}
\def\PY#1#2{\PY@reset\PY@toks#1+\relax+\PY@do{#2}}

\@namedef{PY@tok@w}{\def\PY@tc##1{\textcolor[rgb]{0.73,0.73,0.73}{##1}}}
\@namedef{PY@tok@c}{\let\PY@it=\textit\def\PY@tc##1{\textcolor[rgb]{0.25,0.50,0.50}{##1}}}
\@namedef{PY@tok@cp}{\def\PY@tc##1{\textcolor[rgb]{0.74,0.48,0.00}{##1}}}
\@namedef{PY@tok@k}{\let\PY@bf=\textbf\def\PY@tc##1{\textcolor[rgb]{0.00,0.50,0.00}{##1}}}
\@namedef{PY@tok@kp}{\def\PY@tc##1{\textcolor[rgb]{0.00,0.50,0.00}{##1}}}
\@namedef{PY@tok@kt}{\def\PY@tc##1{\textcolor[rgb]{0.69,0.00,0.25}{##1}}}
\@namedef{PY@tok@o}{\def\PY@tc##1{\textcolor[rgb]{0.40,0.40,0.40}{##1}}}
\@namedef{PY@tok@ow}{\let\PY@bf=\textbf\def\PY@tc##1{\textcolor[rgb]{0.67,0.13,1.00}{##1}}}
\@namedef{PY@tok@nb}{\def\PY@tc##1{\textcolor[rgb]{0.00,0.50,0.00}{##1}}}
\@namedef{PY@tok@nf}{\def\PY@tc##1{\textcolor[rgb]{0.00,0.00,1.00}{##1}}}
\@namedef{PY@tok@nc}{\let\PY@bf=\textbf\def\PY@tc##1{\textcolor[rgb]{0.00,0.00,1.00}{##1}}}
\@namedef{PY@tok@nn}{\let\PY@bf=\textbf\def\PY@tc##1{\textcolor[rgb]{0.00,0.00,1.00}{##1}}}
\@namedef{PY@tok@ne}{\let\PY@bf=\textbf\def\PY@tc##1{\textcolor[rgb]{0.82,0.25,0.23}{##1}}}
\@namedef{PY@tok@nv}{\def\PY@tc##1{\textcolor[rgb]{0.10,0.09,0.49}{##1}}}
\@namedef{PY@tok@no}{\def\PY@tc##1{\textcolor[rgb]{0.53,0.00,0.00}{##1}}}
\@namedef{PY@tok@nl}{\def\PY@tc##1{\textcolor[rgb]{0.63,0.63,0.00}{##1}}}
\@namedef{PY@tok@ni}{\let\PY@bf=\textbf\def\PY@tc##1{\textcolor[rgb]{0.60,0.60,0.60}{##1}}}
\@namedef{PY@tok@na}{\def\PY@tc##1{\textcolor[rgb]{0.49,0.56,0.16}{##1}}}
\@namedef{PY@tok@nt}{\let\PY@bf=\textbf\def\PY@tc##1{\textcolor[rgb]{0.00,0.50,0.00}{##1}}}
\@namedef{PY@tok@nd}{\def\PY@tc##1{\textcolor[rgb]{0.67,0.13,1.00}{##1}}}
\@namedef{PY@tok@s}{\def\PY@tc##1{\textcolor[rgb]{0.73,0.13,0.13}{##1}}}
\@namedef{PY@tok@sd}{\let\PY@it=\textit\def\PY@tc##1{\textcolor[rgb]{0.73,0.13,0.13}{##1}}}
\@namedef{PY@tok@si}{\let\PY@bf=\textbf\def\PY@tc##1{\textcolor[rgb]{0.73,0.40,0.53}{##1}}}
\@namedef{PY@tok@se}{\let\PY@bf=\textbf\def\PY@tc##1{\textcolor[rgb]{0.73,0.40,0.13}{##1}}}
\@namedef{PY@tok@sr}{\def\PY@tc##1{\textcolor[rgb]{0.73,0.40,0.53}{##1}}}
\@namedef{PY@tok@ss}{\def\PY@tc##1{\textcolor[rgb]{0.10,0.09,0.49}{##1}}}
\@namedef{PY@tok@sx}{\def\PY@tc##1{\textcolor[rgb]{0.00,0.50,0.00}{##1}}}
\@namedef{PY@tok@m}{\def\PY@tc##1{\textcolor[rgb]{0.40,0.40,0.40}{##1}}}
\@namedef{PY@tok@gh}{\let\PY@bf=\textbf\def\PY@tc##1{\textcolor[rgb]{0.00,0.00,0.50}{##1}}}
\@namedef{PY@tok@gu}{\let\PY@bf=\textbf\def\PY@tc##1{\textcolor[rgb]{0.50,0.00,0.50}{##1}}}
\@namedef{PY@tok@gd}{\def\PY@tc##1{\textcolor[rgb]{0.63,0.00,0.00}{##1}}}
\@namedef{PY@tok@gi}{\def\PY@tc##1{\textcolor[rgb]{0.00,0.63,0.00}{##1}}}
\@namedef{PY@tok@gr}{\def\PY@tc##1{\textcolor[rgb]{1.00,0.00,0.00}{##1}}}
\@namedef{PY@tok@ge}{\let\PY@it=\textit}
\@namedef{PY@tok@gs}{\let\PY@bf=\textbf}
\@namedef{PY@tok@gp}{\let\PY@bf=\textbf\def\PY@tc##1{\textcolor[rgb]{0.00,0.00,0.50}{##1}}}
\@namedef{PY@tok@go}{\def\PY@tc##1{\textcolor[rgb]{0.53,0.53,0.53}{##1}}}
\@namedef{PY@tok@gt}{\def\PY@tc##1{\textcolor[rgb]{0.00,0.27,0.87}{##1}}}
\@namedef{PY@tok@err}{\def\PY@bc##1{{\setlength{\fboxsep}{\string -\fboxrule}\fcolorbox[rgb]{1.00,0.00,0.00}{1,1,1}{\strut ##1}}}}
\@namedef{PY@tok@kc}{\let\PY@bf=\textbf\def\PY@tc##1{\textcolor[rgb]{0.00,0.50,0.00}{##1}}}
\@namedef{PY@tok@kd}{\let\PY@bf=\textbf\def\PY@tc##1{\textcolor[rgb]{0.00,0.50,0.00}{##1}}}
\@namedef{PY@tok@kn}{\let\PY@bf=\textbf\def\PY@tc##1{\textcolor[rgb]{0.00,0.50,0.00}{##1}}}
\@namedef{PY@tok@kr}{\let\PY@bf=\textbf\def\PY@tc##1{\textcolor[rgb]{0.00,0.50,0.00}{##1}}}
\@namedef{PY@tok@bp}{\def\PY@tc##1{\textcolor[rgb]{0.00,0.50,0.00}{##1}}}
\@namedef{PY@tok@fm}{\def\PY@tc##1{\textcolor[rgb]{0.00,0.00,1.00}{##1}}}
\@namedef{PY@tok@vc}{\def\PY@tc##1{\textcolor[rgb]{0.10,0.09,0.49}{##1}}}
\@namedef{PY@tok@vg}{\def\PY@tc##1{\textcolor[rgb]{0.10,0.09,0.49}{##1}}}
\@namedef{PY@tok@vi}{\def\PY@tc##1{\textcolor[rgb]{0.10,0.09,0.49}{##1}}}
\@namedef{PY@tok@vm}{\def\PY@tc##1{\textcolor[rgb]{0.10,0.09,0.49}{##1}}}
\@namedef{PY@tok@sa}{\def\PY@tc##1{\textcolor[rgb]{0.73,0.13,0.13}{##1}}}
\@namedef{PY@tok@sb}{\def\PY@tc##1{\textcolor[rgb]{0.73,0.13,0.13}{##1}}}
\@namedef{PY@tok@sc}{\def\PY@tc##1{\textcolor[rgb]{0.73,0.13,0.13}{##1}}}
\@namedef{PY@tok@dl}{\def\PY@tc##1{\textcolor[rgb]{0.73,0.13,0.13}{##1}}}
\@namedef{PY@tok@s2}{\def\PY@tc##1{\textcolor[rgb]{0.73,0.13,0.13}{##1}}}
\@namedef{PY@tok@sh}{\def\PY@tc##1{\textcolor[rgb]{0.73,0.13,0.13}{##1}}}
\@namedef{PY@tok@s1}{\def\PY@tc##1{\textcolor[rgb]{0.73,0.13,0.13}{##1}}}
\@namedef{PY@tok@mb}{\def\PY@tc##1{\textcolor[rgb]{0.40,0.40,0.40}{##1}}}
\@namedef{PY@tok@mf}{\def\PY@tc##1{\textcolor[rgb]{0.40,0.40,0.40}{##1}}}
\@namedef{PY@tok@mh}{\def\PY@tc##1{\textcolor[rgb]{0.40,0.40,0.40}{##1}}}
\@namedef{PY@tok@mi}{\def\PY@tc##1{\textcolor[rgb]{0.40,0.40,0.40}{##1}}}
\@namedef{PY@tok@il}{\def\PY@tc##1{\textcolor[rgb]{0.40,0.40,0.40}{##1}}}
\@namedef{PY@tok@mo}{\def\PY@tc##1{\textcolor[rgb]{0.40,0.40,0.40}{##1}}}
\@namedef{PY@tok@ch}{\let\PY@it=\textit\def\PY@tc##1{\textcolor[rgb]{0.25,0.50,0.50}{##1}}}
\@namedef{PY@tok@cm}{\let\PY@it=\textit\def\PY@tc##1{\textcolor[rgb]{0.25,0.50,0.50}{##1}}}
\@namedef{PY@tok@cpf}{\let\PY@it=\textit\def\PY@tc##1{\textcolor[rgb]{0.25,0.50,0.50}{##1}}}
\@namedef{PY@tok@c1}{\let\PY@it=\textit\def\PY@tc##1{\textcolor[rgb]{0.25,0.50,0.50}{##1}}}
\@namedef{PY@tok@cs}{\let\PY@it=\textit\def\PY@tc##1{\textcolor[rgb]{0.25,0.50,0.50}{##1}}}

\def\PYZbs{\char`\\}
\def\PYZus{\char`\_}
\def\PYZob{\char`\{}
\def\PYZcb{\char`\}}
\def\PYZca{\char`\^}
\def\PYZam{\char`\&}
\def\PYZlt{\char`\<}
\def\PYZgt{\char`\>}
\def\PYZsh{\char`\#}
\def\PYZpc{\char`\%}
\def\PYZdl{\char`\$}
\def\PYZhy{\char`\-}
\def\PYZsq{\char`\'}
\def\PYZdq{\char`\"}
\def\PYZti{\char`\~}
% for compatibility with earlier versions
\def\PYZat{@}
\def\PYZlb{[}
\def\PYZrb{]}
\makeatother


    % For linebreaks inside Verbatim environment from package fancyvrb. 
    \makeatletter
        \newbox\Wrappedcontinuationbox 
        \newbox\Wrappedvisiblespacebox 
        \newcommand*\Wrappedvisiblespace {\textcolor{red}{\textvisiblespace}} 
        \newcommand*\Wrappedcontinuationsymbol {\textcolor{red}{\llap{\tiny$\m@th\hookrightarrow$}}} 
        \newcommand*\Wrappedcontinuationindent {3ex } 
        \newcommand*\Wrappedafterbreak {\kern\Wrappedcontinuationindent\copy\Wrappedcontinuationbox} 
        % Take advantage of the already applied Pygments mark-up to insert 
        % potential linebreaks for TeX processing. 
        %        {, <, #, %, $, ' and ": go to next line. 
        %        _, }, ^, &, >, - and ~: stay at end of broken line. 
        % Use of \textquotesingle for straight quote. 
        \newcommand*\Wrappedbreaksatspecials {% 
            \def\PYGZus{\discretionary{\char`\_}{\Wrappedafterbreak}{\char`\_}}% 
            \def\PYGZob{\discretionary{}{\Wrappedafterbreak\char`\{}{\char`\{}}% 
            \def\PYGZcb{\discretionary{\char`\}}{\Wrappedafterbreak}{\char`\}}}% 
            \def\PYGZca{\discretionary{\char`\^}{\Wrappedafterbreak}{\char`\^}}% 
            \def\PYGZam{\discretionary{\char`\&}{\Wrappedafterbreak}{\char`\&}}% 
            \def\PYGZlt{\discretionary{}{\Wrappedafterbreak\char`\<}{\char`\<}}% 
            \def\PYGZgt{\discretionary{\char`\>}{\Wrappedafterbreak}{\char`\>}}% 
            \def\PYGZsh{\discretionary{}{\Wrappedafterbreak\char`\#}{\char`\#}}% 
            \def\PYGZpc{\discretionary{}{\Wrappedafterbreak\char`\%}{\char`\%}}% 
            \def\PYGZdl{\discretionary{}{\Wrappedafterbreak\char`\$}{\char`\$}}% 
            \def\PYGZhy{\discretionary{\char`\-}{\Wrappedafterbreak}{\char`\-}}% 
            \def\PYGZsq{\discretionary{}{\Wrappedafterbreak\textquotesingle}{\textquotesingle}}% 
            \def\PYGZdq{\discretionary{}{\Wrappedafterbreak\char`\"}{\char`\"}}% 
            \def\PYGZti{\discretionary{\char`\~}{\Wrappedafterbreak}{\char`\~}}% 
        } 
        % Some characters . , ; ? ! / are not pygmentized. 
        % This macro makes them "active" and they will insert potential linebreaks 
        \newcommand*\Wrappedbreaksatpunct {% 
            \lccode`\~`\.\lowercase{\def~}{\discretionary{\hbox{\char`\.}}{\Wrappedafterbreak}{\hbox{\char`\.}}}% 
            \lccode`\~`\,\lowercase{\def~}{\discretionary{\hbox{\char`\,}}{\Wrappedafterbreak}{\hbox{\char`\,}}}% 
            \lccode`\~`\;\lowercase{\def~}{\discretionary{\hbox{\char`\;}}{\Wrappedafterbreak}{\hbox{\char`\;}}}% 
            \lccode`\~`\:\lowercase{\def~}{\discretionary{\hbox{\char`\:}}{\Wrappedafterbreak}{\hbox{\char`\:}}}% 
            \lccode`\~`\?\lowercase{\def~}{\discretionary{\hbox{\char`\?}}{\Wrappedafterbreak}{\hbox{\char`\?}}}% 
            \lccode`\~`\!\lowercase{\def~}{\discretionary{\hbox{\char`\!}}{\Wrappedafterbreak}{\hbox{\char`\!}}}% 
            \lccode`\~`\/\lowercase{\def~}{\discretionary{\hbox{\char`\/}}{\Wrappedafterbreak}{\hbox{\char`\/}}}% 
            \catcode`\.\active
            \catcode`\,\active 
            \catcode`\;\active
            \catcode`\:\active
            \catcode`\?\active
            \catcode`\!\active
            \catcode`\/\active 
            \lccode`\~`\~ 	
        }
    \makeatother

    \let\OriginalVerbatim=\Verbatim
    \makeatletter
    \renewcommand{\Verbatim}[1][1]{%
        %\parskip\z@skip
        \sbox\Wrappedcontinuationbox {\Wrappedcontinuationsymbol}%
        \sbox\Wrappedvisiblespacebox {\FV@SetupFont\Wrappedvisiblespace}%
        \def\FancyVerbFormatLine ##1{\hsize\linewidth
            \vtop{\raggedright\hyphenpenalty\z@\exhyphenpenalty\z@
                \doublehyphendemerits\z@\finalhyphendemerits\z@
                \strut ##1\strut}%
        }%
        % If the linebreak is at a space, the latter will be displayed as visible
        % space at end of first line, and a continuation symbol starts next line.
        % Stretch/shrink are however usually zero for typewriter font.
        \def\FV@Space {%
            \nobreak\hskip\z@ plus\fontdimen3\font minus\fontdimen4\font
            \discretionary{\copy\Wrappedvisiblespacebox}{\Wrappedafterbreak}
            {\kern\fontdimen2\font}%
        }%
        
        % Allow breaks at special characters using \PYG... macros.
        \Wrappedbreaksatspecials
        % Breaks at punctuation characters . , ; ? ! and / need catcode=\active 	
        \OriginalVerbatim[#1,codes*=\Wrappedbreaksatpunct]%
    }
    \makeatother

    % Exact colors from NB
    \definecolor{incolor}{HTML}{303F9F}
    \definecolor{outcolor}{HTML}{D84315}
    \definecolor{cellborder}{HTML}{CFCFCF}
    \definecolor{cellbackground}{HTML}{F7F7F7}
    
    % prompt
    \makeatletter
    \newcommand{\boxspacing}{\kern\kvtcb@left@rule\kern\kvtcb@boxsep}
    \makeatother
    \newcommand{\prompt}[4]{
        {\ttfamily\llap{{\color{#2}[#3]:\hspace{3pt}#4}}\vspace{-\baselineskip}}
    }
    

    
    % Prevent overflowing lines due to hard-to-break entities
    \sloppy 
    % Setup hyperref package
    \hypersetup{
      breaklinks=true,  % so long urls are correctly broken across lines
      colorlinks=true,
      urlcolor=urlcolor,
      linkcolor=linkcolor,
      citecolor=citecolor,
      }
    % Slightly bigger margins than the latex defaults
    
    \geometry{verbose,tmargin=1in,bmargin=1in,lmargin=1in,rmargin=1in}
    
    

\begin{document}
    
    \maketitle
    
    

    
    Notes Qiskit UF

    \hypertarget{quantum-states-and-qubits}{%
\section{Quantum States and Qubits}\label{quantum-states-and-qubits}}

    \hypertarget{introduction}{%
\subsection{Introduction}\label{introduction}}

    \hypertarget{the-atoms-of-computation}{%
\subsection{The Atoms of Computation}\label{the-atoms-of-computation}}

    Understanding first classical computation and bits, using the same tools
we will use later for quantum.

    \begin{tcolorbox}[breakable, size=fbox, boxrule=1pt, pad at break*=1mm,colback=cellbackground, colframe=cellborder]
\prompt{In}{incolor}{1}{\boxspacing}
\begin{Verbatim}[commandchars=\\\{\}]
\PY{k+kn}{from} \PY{n+nn}{qiskit} \PY{k+kn}{import} \PY{n}{QuantumCircuit}\PY{p}{,} \PY{n}{assemble}\PY{p}{,} \PY{n}{Aer}
\PY{k+kn}{from} \PY{n+nn}{qiskit}\PY{n+nn}{.}\PY{n+nn}{visualization} \PY{k+kn}{import} \PY{n}{plot\PYZus{}histogram}
\end{Verbatim}
\end{tcolorbox}

    \begin{tcolorbox}[breakable, size=fbox, boxrule=1pt, pad at break*=1mm,colback=cellbackground, colframe=cellborder]
\prompt{In}{incolor}{2}{\boxspacing}
\begin{Verbatim}[commandchars=\\\{\}]
\PY{k+kn}{from} \PY{n+nn}{qiskit\PYZus{}textbook}\PY{n+nn}{.}\PY{n+nn}{widgets} \PY{k+kn}{import} \PY{n}{binary\PYZus{}widget}
\PY{n}{binary\PYZus{}widget}\PY{p}{(}\PY{n}{nbits}\PY{o}{=}\PY{l+m+mi}{5}\PY{p}{)}
\end{Verbatim}
\end{tcolorbox}

    
    \begin{Verbatim}[commandchars=\\\{\}]
VBox(children=(Label(value='Toggle the bits below to change the binary number.'), Label(value='Think of a numb…
    \end{Verbatim}

    
    
    \begin{Verbatim}[commandchars=\\\{\}]
HTML(value='<pre>Binary   Decimal\textbackslash{}n 00000 = 0</pre>')
    \end{Verbatim}

    
    First quantum circuit

    3 jobs:

\begin{itemize}
\tightlist
\item
  enconde the input
\item
  actual computation
\item
  extract output
\end{itemize}

    \begin{tcolorbox}[breakable, size=fbox, boxrule=1pt, pad at break*=1mm,colback=cellbackground, colframe=cellborder]
\prompt{In}{incolor}{3}{\boxspacing}
\begin{Verbatim}[commandchars=\\\{\}]
\PY{c+c1}{\PYZsh{} define QC}
\PY{n}{qc\PYZus{}output} \PY{o}{=} \PY{n}{QuantumCircuit}\PY{p}{(}\PY{l+m+mi}{8}\PY{p}{)} \PY{c+c1}{\PYZsh{} takes n of bits as argument}
\end{Verbatim}
\end{tcolorbox}

    \begin{tcolorbox}[breakable, size=fbox, boxrule=1pt, pad at break*=1mm,colback=cellbackground, colframe=cellborder]
\prompt{In}{incolor}{4}{\boxspacing}
\begin{Verbatim}[commandchars=\\\{\}]
\PY{c+c1}{\PYZsh{} add measurement}
\PY{n}{qc\PYZus{}output}\PY{o}{.}\PY{n}{measure\PYZus{}all}\PY{p}{(}\PY{p}{)} \PY{c+c1}{\PYZsh{} extraction of outputs}
\end{Verbatim}
\end{tcolorbox}

    \begin{tcolorbox}[breakable, size=fbox, boxrule=1pt, pad at break*=1mm,colback=cellbackground, colframe=cellborder]
\prompt{In}{incolor}{5}{\boxspacing}
\begin{Verbatim}[commandchars=\\\{\}]
\PY{n}{qc\PYZus{}output}\PY{o}{.}\PY{n}{draw}\PY{p}{(}\PY{n}{initial\PYZus{}state}\PY{o}{=}\PY{k+kc}{True}\PY{p}{)}
\end{Verbatim}
\end{tcolorbox}
 
            
\prompt{Out}{outcolor}{5}{}
    
    \begin{center}
    \adjustimage{max size={0.9\linewidth}{0.9\paperheight}}{output_11_0.pdf}
    \end{center}
    { \hspace*{\fill} \\}
    

    Notice the qubits are \textbf{always} initialized at \(|0\rangle\).

    \begin{tcolorbox}[breakable, size=fbox, boxrule=1pt, pad at break*=1mm,colback=cellbackground, colframe=cellborder]
\prompt{In}{incolor}{6}{\boxspacing}
\begin{Verbatim}[commandchars=\\\{\}]
\PY{c+c1}{\PYZsh{} simulating the circuit multiple times}
\PY{n}{sim} \PY{o}{=} \PY{n}{Aer}\PY{o}{.}\PY{n}{get\PYZus{}backend}\PY{p}{(}\PY{l+s+s1}{\PYZsq{}}\PY{l+s+s1}{aer\PYZus{}simulator}\PY{l+s+s1}{\PYZsq{}}\PY{p}{)}
\PY{n}{result} \PY{o}{=} \PY{n}{sim}\PY{o}{.}\PY{n}{run}\PY{p}{(}\PY{n}{qc\PYZus{}output}\PY{p}{)}\PY{o}{.}\PY{n}{result}\PY{p}{(}\PY{p}{)}
\PY{n}{count} \PY{o}{=} \PY{n}{result}\PY{o}{.}\PY{n}{get\PYZus{}counts}\PY{p}{(}\PY{p}{)}
\PY{n}{plot\PYZus{}histogram}\PY{p}{(}\PY{n}{count}\PY{p}{)}
\end{Verbatim}
\end{tcolorbox}
 
            
\prompt{Out}{outcolor}{6}{}
    
    \begin{center}
    \adjustimage{max size={0.9\linewidth}{0.9\paperheight}}{output_13_0.pdf}
    \end{center}
    { \hspace*{\fill} \\}
    

    We run many times because quantum computers may have some randomness
when measuring.

Notice this is a \textbf{simulation}, which can be done only to a small
number of qubits \(\approx 30\). To run in a real device, just need to
replace
\texttt{Aer.get\_backend(\textquotesingle{}aer\_simulator\textquotesingle{})}
with the backend of the device.

    \hypertarget{example-creating-an-adder-circuit-item-4-on-textbook}{%
\subsubsection{Example: creating an Adder circuit (item 4 on
textbook)}\label{example-creating-an-adder-circuit-item-4-on-textbook}}

    Remember, we need to:

\begin{itemize}
\tightlist
\item
  enconde the input
\item
  actual computation
\item
  extract output
\end{itemize}

    \hypertarget{enconding-the-input}{%
\paragraph{Enconding the input}\label{enconding-the-input}}

NOT gate first: flips the qubit -\textgreater{} x

    \begin{tcolorbox}[breakable, size=fbox, boxrule=1pt, pad at break*=1mm,colback=cellbackground, colframe=cellborder]
\prompt{In}{incolor}{7}{\boxspacing}
\begin{Verbatim}[commandchars=\\\{\}]
\PY{n}{qc\PYZus{}encode} \PY{o}{=} \PY{n}{QuantumCircuit}\PY{p}{(}\PY{l+m+mi}{8}\PY{p}{)}
\PY{n}{qc\PYZus{}encode}\PY{o}{.}\PY{n}{x}\PY{p}{(}\PY{l+m+mi}{7}\PY{p}{)} \PY{c+c1}{\PYZsh{}(the very last qubit)}
\PY{n}{qc\PYZus{}encode}\PY{o}{.}\PY{n}{draw}\PY{p}{(}\PY{p}{)}
\end{Verbatim}
\end{tcolorbox}
 
            
\prompt{Out}{outcolor}{7}{}
    
    \begin{center}
    \adjustimage{max size={0.9\linewidth}{0.9\paperheight}}{output_18_0.pdf}
    \end{center}
    { \hspace*{\fill} \\}
    

    \begin{tcolorbox}[breakable, size=fbox, boxrule=1pt, pad at break*=1mm,colback=cellbackground, colframe=cellborder]
\prompt{In}{incolor}{8}{\boxspacing}
\begin{Verbatim}[commandchars=\\\{\}]
\PY{n}{qc\PYZus{}encode}\PY{o}{.}\PY{n}{measure\PYZus{}all}\PY{p}{(}\PY{p}{)}
\PY{n}{qc\PYZus{}encode}\PY{o}{.}\PY{n}{draw}\PY{p}{(}\PY{p}{)}
\end{Verbatim}
\end{tcolorbox}
 
            
\prompt{Out}{outcolor}{8}{}
    
    \begin{center}
    \adjustimage{max size={0.9\linewidth}{0.9\paperheight}}{output_19_0.pdf}
    \end{center}
    { \hspace*{\fill} \\}
    

    And similarly to before, we can simulate it:

    \begin{tcolorbox}[breakable, size=fbox, boxrule=1pt, pad at break*=1mm,colback=cellbackground, colframe=cellborder]
\prompt{In}{incolor}{9}{\boxspacing}
\begin{Verbatim}[commandchars=\\\{\}]
\PY{n}{sim} \PY{o}{=} \PY{n}{Aer}\PY{o}{.}\PY{n}{get\PYZus{}backend}\PY{p}{(}\PY{l+s+s1}{\PYZsq{}}\PY{l+s+s1}{aer\PYZus{}simulator}\PY{l+s+s1}{\PYZsq{}}\PY{p}{)}
\PY{n}{result} \PY{o}{=} \PY{n}{sim}\PY{o}{.}\PY{n}{run}\PY{p}{(}\PY{n}{qc\PYZus{}encode}\PY{p}{)}\PY{o}{.}\PY{n}{result}\PY{p}{(}\PY{p}{)}
\PY{n}{counts} \PY{o}{=} \PY{n}{result}\PY{o}{.}\PY{n}{get\PYZus{}counts}\PY{p}{(}\PY{p}{)}
\PY{n}{plot\PYZus{}histogram}\PY{p}{(}\PY{n}{counts}\PY{p}{)}
\end{Verbatim}
\end{tcolorbox}
 
            
\prompt{Out}{outcolor}{9}{}
    
    \begin{center}
    \adjustimage{max size={0.9\linewidth}{0.9\paperheight}}{output_21_0.pdf}
    \end{center}
    { \hspace*{\fill} \\}
    

    \textbf{Notice it reads from right to left}, to be similar to the
representation of numbers in the decimal system

\[
1 \times 2^7 + 0 \times 2^6 + 0 \times 2^5 + 0 \times 2^4 + 0 \times 2^3 + 0 \times 2^2+ 0 \times 2^1 + 0 \times 2^0
=10000000
\]

    \begin{tcolorbox}[breakable, size=fbox, boxrule=1pt, pad at break*=1mm,colback=cellbackground, colframe=cellborder]
\prompt{In}{incolor}{10}{\boxspacing}
\begin{Verbatim}[commandchars=\\\{\}]
\PY{n}{qc\PYZus{}encode} \PY{o}{=} \PY{n}{QuantumCircuit}\PY{p}{(}\PY{l+m+mi}{8}\PY{p}{)}
\PY{n}{qc\PYZus{}encode}\PY{o}{.}\PY{n}{x}\PY{p}{(}\PY{l+m+mi}{1}\PY{p}{)}
\PY{n}{qc\PYZus{}encode}\PY{o}{.}\PY{n}{x}\PY{p}{(}\PY{l+m+mi}{5}\PY{p}{)}

\PY{n}{qc\PYZus{}encode}\PY{o}{.}\PY{n}{draw}\PY{p}{(}\PY{p}{)}
\end{Verbatim}
\end{tcolorbox}
 
            
\prompt{Out}{outcolor}{10}{}
    
    \begin{center}
    \adjustimage{max size={0.9\linewidth}{0.9\paperheight}}{output_23_0.pdf}
    \end{center}
    { \hspace*{\fill} \\}
    

    Remember the usual addition, carry one algorithm.

    \begin{eqnarray}
&  10 \\
+&  01 \\
=&  11
\end{eqnarray}

    The sums in decimal then becomes,

\begin{eqnarray}
0+0 &= 00 \\
0+1 &= 01 \\
1+0 &= 01 \\
1+1 &= 10 
\end{eqnarray}

which is called a \textbf{half adder}.

    

    The two qubits to add are encoded in \(0\) and \(1\). In the above
circuit, we are looking for the solution of \(1+1\). The results will be
stored on the qubits 2 and 3 and will store in classical bits 0 and 1,
respectively.

\emph{Dashed lines are made with the} \texttt{barrier} \emph{command}.

    Remember

\begin{eqnarray}
0+0 &= 00 \\
0+1 &= 01 \\
1+0 &= 01 \\
1+1 &= 10 
\end{eqnarray}

Notice this seems like \textbf{XOR gate}:

\begin{eqnarray}
0+0 &= 0 \\
0+1 &= 1 \\
1+0 &= 1 \\
1+1 &= 0 
\end{eqnarray}

which in quantum computers is done by the \textbf{CNOT gate},
\texttt{cx}.

    \begin{tcolorbox}[breakable, size=fbox, boxrule=1pt, pad at break*=1mm,colback=cellbackground, colframe=cellborder]
\prompt{In}{incolor}{11}{\boxspacing}
\begin{Verbatim}[commandchars=\\\{\}]
\PY{n}{qc\PYZus{}cnot} \PY{o}{=} \PY{n}{QuantumCircuit}\PY{p}{(}\PY{l+m+mi}{2}\PY{p}{)}
\PY{n}{qc\PYZus{}cnot}\PY{o}{.}\PY{n}{cx}\PY{p}{(}\PY{l+m+mi}{0}\PY{p}{,}\PY{l+m+mi}{1}\PY{p}{)} \PY{c+c1}{\PYZsh{}\PYZsh{} first is the control, second is the target}
\PY{n}{qc\PYZus{}cnot}\PY{o}{.}\PY{n}{draw}\PY{p}{(}\PY{p}{)}
\end{Verbatim}
\end{tcolorbox}
 
            
\prompt{Out}{outcolor}{11}{}
    
    \begin{center}
    \adjustimage{max size={0.9\linewidth}{0.9\paperheight}}{output_30_0.pdf}
    \end{center}
    { \hspace*{\fill} \\}
    

    It compares both inputs to see if they are different, and overwrites the
target bit with the answer. Or, said in another manner, it flips the
target if the control is \(1\).

Trying it:

    \begin{tcolorbox}[breakable, size=fbox, boxrule=1pt, pad at break*=1mm,colback=cellbackground, colframe=cellborder]
\prompt{In}{incolor}{12}{\boxspacing}
\begin{Verbatim}[commandchars=\\\{\}]
\PY{n}{qc} \PY{o}{=} \PY{n}{QuantumCircuit}\PY{p}{(}\PY{l+m+mi}{2}\PY{p}{,}\PY{l+m+mi}{2}\PY{p}{)} \PY{c+c1}{\PYZsh{} Two quantum, two classical regimes}
\PY{n}{qc}\PY{o}{.}\PY{n}{draw}\PY{p}{(}\PY{p}{)}
\end{Verbatim}
\end{tcolorbox}
 
            
\prompt{Out}{outcolor}{12}{}
    
    \begin{center}
    \adjustimage{max size={0.9\linewidth}{0.9\paperheight}}{output_32_0.pdf}
    \end{center}
    { \hspace*{\fill} \\}
    

    \begin{tcolorbox}[breakable, size=fbox, boxrule=1pt, pad at break*=1mm,colback=cellbackground, colframe=cellborder]
\prompt{In}{incolor}{13}{\boxspacing}
\begin{Verbatim}[commandchars=\\\{\}]
\PY{n}{qc}\PY{o}{.}\PY{n}{x}\PY{p}{(}\PY{l+m+mi}{0}\PY{p}{)}
\PY{n}{qc}\PY{o}{.}\PY{n}{draw}\PY{p}{(}\PY{p}{)}
\end{Verbatim}
\end{tcolorbox}
 
            
\prompt{Out}{outcolor}{13}{}
    
    \begin{center}
    \adjustimage{max size={0.9\linewidth}{0.9\paperheight}}{output_33_0.pdf}
    \end{center}
    { \hspace*{\fill} \\}
    

    \begin{tcolorbox}[breakable, size=fbox, boxrule=1pt, pad at break*=1mm,colback=cellbackground, colframe=cellborder]
\prompt{In}{incolor}{14}{\boxspacing}
\begin{Verbatim}[commandchars=\\\{\}]
\PY{n}{qc}\PY{o}{.}\PY{n}{cx}\PY{p}{(}\PY{l+m+mi}{0}\PY{p}{,}\PY{l+m+mi}{1}\PY{p}{)}
\PY{n}{qc}\PY{o}{.}\PY{n}{draw}\PY{p}{(}\PY{p}{)}
\end{Verbatim}
\end{tcolorbox}
 
            
\prompt{Out}{outcolor}{14}{}
    
    \begin{center}
    \adjustimage{max size={0.9\linewidth}{0.9\paperheight}}{output_34_0.pdf}
    \end{center}
    { \hspace*{\fill} \\}
    

    \begin{tcolorbox}[breakable, size=fbox, boxrule=1pt, pad at break*=1mm,colback=cellbackground, colframe=cellborder]
\prompt{In}{incolor}{15}{\boxspacing}
\begin{Verbatim}[commandchars=\\\{\}]
\PY{n}{qc}\PY{o}{.}\PY{n}{measure}\PY{p}{(}\PY{l+m+mi}{0}\PY{p}{,}\PY{l+m+mi}{0}\PY{p}{)}
\PY{n}{qc}\PY{o}{.}\PY{n}{measure}\PY{p}{(}\PY{l+m+mi}{1}\PY{p}{,}\PY{l+m+mi}{1}\PY{p}{)}
\PY{n}{qc}\PY{o}{.}\PY{n}{draw}\PY{p}{(}\PY{p}{)}
\end{Verbatim}
\end{tcolorbox}
 
            
\prompt{Out}{outcolor}{15}{}
    
    \begin{center}
    \adjustimage{max size={0.9\linewidth}{0.9\paperheight}}{output_35_0.pdf}
    \end{center}
    { \hspace*{\fill} \\}
    

    The result should be \(11\)

    \begin{tcolorbox}[breakable, size=fbox, boxrule=1pt, pad at break*=1mm,colback=cellbackground, colframe=cellborder]
\prompt{In}{incolor}{16}{\boxspacing}
\begin{Verbatim}[commandchars=\\\{\}]
\PY{n}{sim} \PY{o}{=} \PY{n}{Aer}\PY{o}{.}\PY{n}{get\PYZus{}backend}\PY{p}{(}\PY{l+s+s1}{\PYZsq{}}\PY{l+s+s1}{aer\PYZus{}simulator}\PY{l+s+s1}{\PYZsq{}}\PY{p}{)}
\PY{n}{result} \PY{o}{=} \PY{n}{sim}\PY{o}{.}\PY{n}{run}\PY{p}{(}\PY{n}{qc}\PY{p}{)}\PY{o}{.}\PY{n}{result}\PY{p}{(}\PY{p}{)}
\PY{n}{counts} \PY{o}{=} \PY{n}{result}\PY{o}{.}\PY{n}{get\PYZus{}counts}\PY{p}{(}\PY{p}{)}
\PY{n}{plot\PYZus{}histogram}\PY{p}{(}\PY{n}{counts}\PY{p}{)}
\end{Verbatim}
\end{tcolorbox}
 
            
\prompt{Out}{outcolor}{16}{}
    
    \begin{center}
    \adjustimage{max size={0.9\linewidth}{0.9\paperheight}}{output_37_0.pdf}
    \end{center}
    { \hspace*{\fill} \\}
    

    Or, all in one single cell:

    \begin{tcolorbox}[breakable, size=fbox, boxrule=1pt, pad at break*=1mm,colback=cellbackground, colframe=cellborder]
\prompt{In}{incolor}{17}{\boxspacing}
\begin{Verbatim}[commandchars=\\\{\}]
\PY{n}{qc} \PY{o}{=} \PY{n}{QuantumCircuit}\PY{p}{(}\PY{l+m+mi}{2}\PY{p}{,}\PY{l+m+mi}{2}\PY{p}{)}
\PY{n}{qc}\PY{o}{.}\PY{n}{x}\PY{p}{(}\PY{l+m+mi}{0}\PY{p}{)}
\PY{n}{qc}\PY{o}{.}\PY{n}{cx}\PY{p}{(}\PY{l+m+mi}{0}\PY{p}{,}\PY{l+m+mi}{1}\PY{p}{)}
\PY{n}{qc}\PY{o}{.}\PY{n}{measure}\PY{p}{(}\PY{l+m+mi}{0}\PY{p}{,}\PY{l+m+mi}{0}\PY{p}{)}
\PY{n}{qc}\PY{o}{.}\PY{n}{measure}\PY{p}{(}\PY{l+m+mi}{1}\PY{p}{,}\PY{l+m+mi}{1}\PY{p}{)}
\PY{n}{qc}\PY{o}{.}\PY{n}{draw}\PY{p}{(}\PY{p}{)}
\end{Verbatim}
\end{tcolorbox}
 
            
\prompt{Out}{outcolor}{17}{}
    
    \begin{center}
    \adjustimage{max size={0.9\linewidth}{0.9\paperheight}}{output_39_0.pdf}
    \end{center}
    { \hspace*{\fill} \\}
    

    For our half adder, we don't want to overwrite one of our inputs.

Instead, \emph{we want to write the result on a different pair of
qubits}. For this, we can use two CNOTs.

    \begin{tcolorbox}[breakable, size=fbox, boxrule=1pt, pad at break*=1mm,colback=cellbackground, colframe=cellborder]
\prompt{In}{incolor}{18}{\boxspacing}
\begin{Verbatim}[commandchars=\\\{\}]
\PY{n}{qc\PYZus{}ha} \PY{o}{=} \PY{n}{QuantumCircuit}\PY{p}{(}\PY{l+m+mi}{4}\PY{p}{,}\PY{l+m+mi}{2}\PY{p}{)}
\PY{c+c1}{\PYZsh{} encode inputs in qubits 0 and 1}
\PY{n}{qc\PYZus{}ha}\PY{o}{.}\PY{n}{x}\PY{p}{(}\PY{l+m+mi}{0}\PY{p}{)} \PY{c+c1}{\PYZsh{} For a=0, remove this line. For a=1, leave it.}
\PY{n}{qc\PYZus{}ha}\PY{o}{.}\PY{n}{x}\PY{p}{(}\PY{l+m+mi}{1}\PY{p}{)} \PY{c+c1}{\PYZsh{} For b=0, remove this line. For b=1, l}
\PY{n}{qc\PYZus{}ha}\PY{o}{.}\PY{n}{draw}\PY{p}{(}\PY{p}{)}
\end{Verbatim}
\end{tcolorbox}
 
            
\prompt{Out}{outcolor}{18}{}
    
    \begin{center}
    \adjustimage{max size={0.9\linewidth}{0.9\paperheight}}{output_41_0.pdf}
    \end{center}
    { \hspace*{\fill} \\}
    

    \begin{tcolorbox}[breakable, size=fbox, boxrule=1pt, pad at break*=1mm,colback=cellbackground, colframe=cellborder]
\prompt{In}{incolor}{19}{\boxspacing}
\begin{Verbatim}[commandchars=\\\{\}]
\PY{n}{qc\PYZus{}ha}\PY{o}{.}\PY{n}{barrier}\PY{p}{(}\PY{p}{)}
\PY{c+c1}{\PYZsh{} use cnots to write the XOR of the inputs on qubit 2}
\PY{n}{qc\PYZus{}ha}\PY{o}{.}\PY{n}{cx}\PY{p}{(}\PY{l+m+mi}{0}\PY{p}{,}\PY{l+m+mi}{2}\PY{p}{)}
\PY{n}{qc\PYZus{}ha}\PY{o}{.}\PY{n}{cx}\PY{p}{(}\PY{l+m+mi}{1}\PY{p}{,}\PY{l+m+mi}{2}\PY{p}{)}
\PY{n}{qc\PYZus{}ha}\PY{o}{.}\PY{n}{barrier}\PY{p}{(}\PY{p}{)}
\PY{n}{qc\PYZus{}ha}\PY{o}{.}\PY{n}{draw}\PY{p}{(}\PY{p}{)}
\end{Verbatim}
\end{tcolorbox}
 
            
\prompt{Out}{outcolor}{19}{}
    
    \begin{center}
    \adjustimage{max size={0.9\linewidth}{0.9\paperheight}}{output_42_0.pdf}
    \end{center}
    { \hspace*{\fill} \\}
    

    \begin{tcolorbox}[breakable, size=fbox, boxrule=1pt, pad at break*=1mm,colback=cellbackground, colframe=cellborder]
\prompt{In}{incolor}{20}{\boxspacing}
\begin{Verbatim}[commandchars=\\\{\}]
\PY{c+c1}{\PYZsh{} extract outputs}
\PY{n}{qc\PYZus{}ha}\PY{o}{.}\PY{n}{measure}\PY{p}{(}\PY{l+m+mi}{2}\PY{p}{,}\PY{l+m+mi}{0}\PY{p}{)} \PY{c+c1}{\PYZsh{} extract XOR value}
\PY{n}{qc\PYZus{}ha}\PY{o}{.}\PY{n}{measure}\PY{p}{(}\PY{l+m+mi}{3}\PY{p}{,}\PY{l+m+mi}{1}\PY{p}{)}

\PY{n}{qc\PYZus{}ha}\PY{o}{.}\PY{n}{draw}\PY{p}{(}\PY{p}{)}
\end{Verbatim}
\end{tcolorbox}
 
            
\prompt{Out}{outcolor}{20}{}
    
    \begin{center}
    \adjustimage{max size={0.9\linewidth}{0.9\paperheight}}{output_43_0.pdf}
    \end{center}
    { \hspace*{\fill} \\}
    

    All in one cell:

    \begin{tcolorbox}[breakable, size=fbox, boxrule=1pt, pad at break*=1mm,colback=cellbackground, colframe=cellborder]
\prompt{In}{incolor}{21}{\boxspacing}
\begin{Verbatim}[commandchars=\\\{\}]
\PY{n}{qc\PYZus{}ha} \PY{o}{=} \PY{n}{QuantumCircuit}\PY{p}{(}\PY{l+m+mi}{4}\PY{p}{,}\PY{l+m+mi}{2}\PY{p}{)}
\PY{c+c1}{\PYZsh{} encode inputs in qubits 0 and 1}
\PY{n}{qc\PYZus{}ha}\PY{o}{.}\PY{n}{x}\PY{p}{(}\PY{l+m+mi}{0}\PY{p}{)} \PY{c+c1}{\PYZsh{} For a=0, remove this line. For a=1, leave it.}
\PY{n}{qc\PYZus{}ha}\PY{o}{.}\PY{n}{x}\PY{p}{(}\PY{l+m+mi}{1}\PY{p}{)} \PY{c+c1}{\PYZsh{} For b=0, remove this line. For b=1, leave it.}
\PY{n}{qc\PYZus{}ha}\PY{o}{.}\PY{n}{barrier}\PY{p}{(}\PY{p}{)}
\PY{c+c1}{\PYZsh{} use cnots to write the XOR of the inputs on qubit 2}
\PY{n}{qc\PYZus{}ha}\PY{o}{.}\PY{n}{cx}\PY{p}{(}\PY{l+m+mi}{0}\PY{p}{,}\PY{l+m+mi}{2}\PY{p}{)}
\PY{n}{qc\PYZus{}ha}\PY{o}{.}\PY{n}{cx}\PY{p}{(}\PY{l+m+mi}{1}\PY{p}{,}\PY{l+m+mi}{2}\PY{p}{)}
\PY{n}{qc\PYZus{}ha}\PY{o}{.}\PY{n}{barrier}\PY{p}{(}\PY{p}{)}
\PY{c+c1}{\PYZsh{} extract outputs}
\PY{n}{qc\PYZus{}ha}\PY{o}{.}\PY{n}{measure}\PY{p}{(}\PY{l+m+mi}{2}\PY{p}{,}\PY{l+m+mi}{0}\PY{p}{)} \PY{c+c1}{\PYZsh{} extract XOR value}
\PY{n}{qc\PYZus{}ha}\PY{o}{.}\PY{n}{measure}\PY{p}{(}\PY{l+m+mi}{3}\PY{p}{,}\PY{l+m+mi}{1}\PY{p}{)}

\PY{n}{qc\PYZus{}ha}\PY{o}{.}\PY{n}{draw}\PY{p}{(}\PY{p}{)}
\end{Verbatim}
\end{tcolorbox}
 
            
\prompt{Out}{outcolor}{21}{}
    
    \begin{center}
    \adjustimage{max size={0.9\linewidth}{0.9\paperheight}}{output_45_0.pdf}
    \end{center}
    { \hspace*{\fill} \\}
    

    So, \(q_2\) has the result of the first bit.

For the second qubit, recorded in \(q3\), it will only be \(1\) when
\(1+1 = 10\). Therefore, we can check when both inputs are \(1\). If
both are, we need a \textbf{NOT} gate on \(q3\), controlled on both
\(q1\) and \(q2\): \textbf{Toffoli gate} (basically an AND gate);
\texttt{ccx}.

Repeating the circuit above:

    \begin{tcolorbox}[breakable, size=fbox, boxrule=1pt, pad at break*=1mm,colback=cellbackground, colframe=cellborder]
\prompt{In}{incolor}{22}{\boxspacing}
\begin{Verbatim}[commandchars=\\\{\}]
\PY{n}{qc\PYZus{}ha} \PY{o}{=} \PY{n}{QuantumCircuit}\PY{p}{(}\PY{l+m+mi}{4}\PY{p}{,}\PY{l+m+mi}{2}\PY{p}{)}
\PY{c+c1}{\PYZsh{} encode inputs in qubits 0 and 1}
\PY{n}{qc\PYZus{}ha}\PY{o}{.}\PY{n}{x}\PY{p}{(}\PY{l+m+mi}{0}\PY{p}{)} \PY{c+c1}{\PYZsh{} For a=0, remove the this line. For a=1, leave it.}
\PY{n}{qc\PYZus{}ha}\PY{o}{.}\PY{n}{x}\PY{p}{(}\PY{l+m+mi}{1}\PY{p}{)} \PY{c+c1}{\PYZsh{} For b=0, remove the this line. For b=1, leave it.}
\PY{n}{qc\PYZus{}ha}\PY{o}{.}\PY{n}{barrier}\PY{p}{(}\PY{p}{)}
\PY{c+c1}{\PYZsh{} use cnots to write the XOR of the inputs on qubit 2}
\PY{n}{qc\PYZus{}ha}\PY{o}{.}\PY{n}{cx}\PY{p}{(}\PY{l+m+mi}{0}\PY{p}{,}\PY{l+m+mi}{2}\PY{p}{)}
\PY{n}{qc\PYZus{}ha}\PY{o}{.}\PY{n}{cx}\PY{p}{(}\PY{l+m+mi}{1}\PY{p}{,}\PY{l+m+mi}{2}\PY{p}{)}
\PY{n}{qc\PYZus{}ha}\PY{o}{.}\PY{n}{draw}\PY{p}{(}\PY{p}{)}
\end{Verbatim}
\end{tcolorbox}
 
            
\prompt{Out}{outcolor}{22}{}
    
    \begin{center}
    \adjustimage{max size={0.9\linewidth}{0.9\paperheight}}{output_47_0.pdf}
    \end{center}
    { \hspace*{\fill} \\}
    

    \begin{tcolorbox}[breakable, size=fbox, boxrule=1pt, pad at break*=1mm,colback=cellbackground, colframe=cellborder]
\prompt{In}{incolor}{23}{\boxspacing}
\begin{Verbatim}[commandchars=\\\{\}]
\PY{c+c1}{\PYZsh{} use ccx to write the AND of the inputs on qubit 3}
\PY{n}{qc\PYZus{}ha}\PY{o}{.}\PY{n}{ccx}\PY{p}{(}\PY{l+m+mi}{0}\PY{p}{,}\PY{l+m+mi}{1}\PY{p}{,}\PY{l+m+mi}{3}\PY{p}{)}
\PY{n}{qc\PYZus{}ha}\PY{o}{.}\PY{n}{barrier}\PY{p}{(}\PY{p}{)}
\PY{c+c1}{\PYZsh{} extract outputs}
\PY{n}{qc\PYZus{}ha}\PY{o}{.}\PY{n}{measure}\PY{p}{(}\PY{l+m+mi}{2}\PY{p}{,}\PY{l+m+mi}{0}\PY{p}{)} \PY{c+c1}{\PYZsh{} extract XOR value}
\PY{n}{qc\PYZus{}ha}\PY{o}{.}\PY{n}{measure}\PY{p}{(}\PY{l+m+mi}{3}\PY{p}{,}\PY{l+m+mi}{1}\PY{p}{)} \PY{c+c1}{\PYZsh{} extract AND value}

\PY{n}{qc\PYZus{}ha}\PY{o}{.}\PY{n}{draw}\PY{p}{(}\PY{p}{)}
\end{Verbatim}
\end{tcolorbox}
 
            
\prompt{Out}{outcolor}{23}{}
    
    \begin{center}
    \adjustimage{max size={0.9\linewidth}{0.9\paperheight}}{output_48_0.pdf}
    \end{center}
    { \hspace*{\fill} \\}
    

    Now let's calculate the outcomes of this circuit. Notice that so far we
have only created the circuit. Now let's calculate it:

    \begin{tcolorbox}[breakable, size=fbox, boxrule=1pt, pad at break*=1mm,colback=cellbackground, colframe=cellborder]
\prompt{In}{incolor}{24}{\boxspacing}
\begin{Verbatim}[commandchars=\\\{\}]
\PY{c+c1}{\PYZsh{}\PYZsh{}\PYZsh{}\PYZsh{} Notice that assemble \PYZhy{}\PYZgt{} qobj is not needed anymore, }
\PY{c+c1}{\PYZsh{}\PYZsh{}\PYZsh{}\PYZsh{} can run the circuit directly}
\PY{c+c1}{\PYZsh{}qobj = assemble(qc\PYZus{}ha)}
\PY{c+c1}{\PYZsh{}counts = sim.run(qobj).result().get\PYZus{}counts()}

\PY{n}{counts} \PY{o}{=} \PY{n}{sim}\PY{o}{.}\PY{n}{run}\PY{p}{(}\PY{n}{qc\PYZus{}ha}\PY{p}{)}\PY{o}{.}\PY{n}{result}\PY{p}{(}\PY{p}{)}\PY{o}{.}\PY{n}{get\PYZus{}counts}\PY{p}{(}\PY{p}{)}
\PY{n}{plot\PYZus{}histogram}\PY{p}{(}\PY{n}{counts}\PY{p}{)}
\end{Verbatim}
\end{tcolorbox}
 
            
\prompt{Out}{outcolor}{24}{}
    
    \begin{center}
    \adjustimage{max size={0.9\linewidth}{0.9\paperheight}}{output_50_0.pdf}
    \end{center}
    { \hspace*{\fill} \\}
    

    \begin{tcolorbox}[breakable, size=fbox, boxrule=1pt, pad at break*=1mm,colback=cellbackground, colframe=cellborder]
\prompt{In}{incolor}{25}{\boxspacing}
\begin{Verbatim}[commandchars=\\\{\}]
\PY{c+c1}{\PYZsh{}\PYZsh{} trying all possibilities:}
\PY{k}{for} \PY{n}{q0}\PY{p}{,} \PY{n}{q1} \PY{o+ow}{in} \PY{p}{[}\PY{p}{(}\PY{n}{q0}\PY{p}{,}\PY{n}{q1}\PY{p}{)} \PY{k}{for} \PY{n}{q0} \PY{o+ow}{in} \PY{p}{[}\PY{l+m+mi}{0}\PY{p}{,}\PY{l+m+mi}{1}\PY{p}{]} \PY{k}{for} \PY{n}{q1} \PY{o+ow}{in} \PY{p}{[}\PY{l+m+mi}{0}\PY{p}{,}\PY{l+m+mi}{1}\PY{p}{]}\PY{p}{]}\PY{p}{:}
    \PY{n+nb}{print}\PY{p}{(}\PY{n}{q0}\PY{p}{,}\PY{n}{q1}\PY{p}{)}
    \PY{n}{qc\PYZus{}ha} \PY{o}{=} \PY{n}{QuantumCircuit}\PY{p}{(}\PY{l+m+mi}{4}\PY{p}{,}\PY{l+m+mi}{2}\PY{p}{)}
    \PY{c+c1}{\PYZsh{} encode inputs in qubits 0 and 1}
    \PY{k}{if} \PY{n}{q0} \PY{o}{==} \PY{l+m+mi}{1}\PY{p}{:}
        \PY{n}{qc\PYZus{}ha}\PY{o}{.}\PY{n}{x}\PY{p}{(}\PY{l+m+mi}{0}\PY{p}{)} \PY{c+c1}{\PYZsh{} For a=0, remove the this line. For a=1, leave it.}
    \PY{k}{if} \PY{n}{q1} \PY{o}{==} \PY{l+m+mi}{1}\PY{p}{:}
        \PY{n}{qc\PYZus{}ha}\PY{o}{.}\PY{n}{x}\PY{p}{(}\PY{l+m+mi}{1}\PY{p}{)} \PY{c+c1}{\PYZsh{} For b=0, remove the this line. For b=1, leave it.}
    \PY{n}{qc\PYZus{}ha}\PY{o}{.}\PY{n}{barrier}\PY{p}{(}\PY{p}{)}
    \PY{c+c1}{\PYZsh{} use cnots to write the XOR of the inputs on qubit 2}
    \PY{n}{qc\PYZus{}ha}\PY{o}{.}\PY{n}{cx}\PY{p}{(}\PY{l+m+mi}{0}\PY{p}{,}\PY{l+m+mi}{2}\PY{p}{)}
    \PY{n}{qc\PYZus{}ha}\PY{o}{.}\PY{n}{cx}\PY{p}{(}\PY{l+m+mi}{1}\PY{p}{,}\PY{l+m+mi}{2}\PY{p}{)}
    \PY{c+c1}{\PYZsh{} use ccx to write the AND of the inputs on qubit 3}
    \PY{n}{qc\PYZus{}ha}\PY{o}{.}\PY{n}{ccx}\PY{p}{(}\PY{l+m+mi}{0}\PY{p}{,}\PY{l+m+mi}{1}\PY{p}{,}\PY{l+m+mi}{3}\PY{p}{)}
    \PY{n}{qc\PYZus{}ha}\PY{o}{.}\PY{n}{barrier}\PY{p}{(}\PY{p}{)}
    \PY{c+c1}{\PYZsh{} extract outputs}
    \PY{n}{qc\PYZus{}ha}\PY{o}{.}\PY{n}{measure}\PY{p}{(}\PY{l+m+mi}{2}\PY{p}{,}\PY{l+m+mi}{0}\PY{p}{)} \PY{c+c1}{\PYZsh{} extract XOR value}
    \PY{n}{qc\PYZus{}ha}\PY{o}{.}\PY{n}{measure}\PY{p}{(}\PY{l+m+mi}{3}\PY{p}{,}\PY{l+m+mi}{1}\PY{p}{)} \PY{c+c1}{\PYZsh{} extract AND value}

    \PY{n}{qc\PYZus{}ha}\PY{o}{.}\PY{n}{draw}\PY{p}{(}\PY{p}{)}
    
    \PY{n}{counts} \PY{o}{=} \PY{n}{sim}\PY{o}{.}\PY{n}{run}\PY{p}{(}\PY{n}{qc\PYZus{}ha}\PY{p}{)}\PY{o}{.}\PY{n}{result}\PY{p}{(}\PY{p}{)}\PY{o}{.}\PY{n}{get\PYZus{}counts}\PY{p}{(}\PY{p}{)}
    \PY{n}{display}\PY{p}{(}\PY{n}{plot\PYZus{}histogram}\PY{p}{(}\PY{n}{counts}\PY{p}{,} \PY{n}{figsize}\PY{o}{=}\PY{p}{(}\PY{l+m+mi}{4}\PY{p}{,} \PY{l+m+mi}{2}\PY{p}{)}\PY{p}{)}\PY{p}{)}
\end{Verbatim}
\end{tcolorbox}

    \begin{Verbatim}[commandchars=\\\{\}]
0 0
    \end{Verbatim}

    \begin{center}
    \adjustimage{max size={0.9\linewidth}{0.9\paperheight}}{output_51_1.pdf}
    \end{center}
    { \hspace*{\fill} \\}
    
    \begin{Verbatim}[commandchars=\\\{\}]
0 1
    \end{Verbatim}

    \begin{center}
    \adjustimage{max size={0.9\linewidth}{0.9\paperheight}}{output_51_3.pdf}
    \end{center}
    { \hspace*{\fill} \\}
    
    \begin{Verbatim}[commandchars=\\\{\}]
1 0
    \end{Verbatim}

    \begin{center}
    \adjustimage{max size={0.9\linewidth}{0.9\paperheight}}{output_51_5.pdf}
    \end{center}
    { \hspace*{\fill} \\}
    
    \begin{Verbatim}[commandchars=\\\{\}]
1 1
    \end{Verbatim}

    \begin{center}
    \adjustimage{max size={0.9\linewidth}{0.9\paperheight}}{output_51_7.pdf}
    \end{center}
    { \hspace*{\fill} \\}
    
    The half-adder contains everything needed for addition!

NOT+CNOT+Toffoli: can add any set of numbers of any size.

In fact, we could even do without the CNOT and NOT (only used to go from
\(0 \rightarrow 1\)). \textbf{The Toffoli gate is the atom of quantum
computation.}

    \hypertarget{representing-qubit-states}{%
\subsection{Representing Qubit States}\label{representing-qubit-states}}

    Example of statevector

\[ | q_0 \rangle 
= 
\begin{bmatrix}
\frac{1}{\sqrt{2}} \\
\frac{i}{\sqrt{2}} 
\end{bmatrix}
\]

and since \(|0\rangle\) and \(|1\rangle\) form and orthonormal basis, we
can write the statevector on this basis as a \textbf{superposition} of
\(|0\rangle\) and \(|1\rangle\):

\[
| q_0 \rangle 
= \frac{1}{\sqrt{2}}|0\rangle + \frac{i}{\sqrt{2}}|1\rangle
\]

    \hypertarget{exploring-qubits-with-qiskit}{%
\subsubsection{Exploring Qubits with
Qiskit}\label{exploring-qubits-with-qiskit}}

    \begin{tcolorbox}[breakable, size=fbox, boxrule=1pt, pad at break*=1mm,colback=cellbackground, colframe=cellborder]
\prompt{In}{incolor}{26}{\boxspacing}
\begin{Verbatim}[commandchars=\\\{\}]
\PY{k+kn}{from} \PY{n+nn}{qiskit} \PY{k+kn}{import} \PY{n}{QuantumCircuit}\PY{p}{,} \PY{n}{Aer}
\PY{k+kn}{from} \PY{n+nn}{qiskit}\PY{n+nn}{.}\PY{n+nn}{visualization} \PY{k+kn}{import} \PY{n}{plot\PYZus{}histogram}\PY{p}{,} \PY{n}{plot\PYZus{}bloch\PYZus{}vector}
\PY{k+kn}{from} \PY{n+nn}{math} \PY{k+kn}{import} \PY{n}{sqrt}\PY{p}{,} \PY{n}{pi}
\end{Verbatim}
\end{tcolorbox}

    \begin{tcolorbox}[breakable, size=fbox, boxrule=1pt, pad at break*=1mm,colback=cellbackground, colframe=cellborder]
\prompt{In}{incolor}{27}{\boxspacing}
\begin{Verbatim}[commandchars=\\\{\}]
\PY{n}{qc} \PY{o}{=} \PY{n}{QuantumCircuit}\PY{p}{(}\PY{l+m+mi}{1}\PY{p}{)}\PY{c+c1}{\PYZsh{} Create a quantum circuit with one qubit}
\end{Verbatim}
\end{tcolorbox}

    Qubits always start on \(|0\rangle\), but we can use the
\texttt{initialize()} method to transform it.

    \begin{tcolorbox}[breakable, size=fbox, boxrule=1pt, pad at break*=1mm,colback=cellbackground, colframe=cellborder]
\prompt{In}{incolor}{28}{\boxspacing}
\begin{Verbatim}[commandchars=\\\{\}]
\PY{n}{qc} \PY{o}{=} \PY{n}{QuantumCircuit}\PY{p}{(}\PY{l+m+mi}{1}\PY{p}{)}\PY{c+c1}{\PYZsh{} Create a quantum circuit with one qubit}
\PY{n}{initial\PYZus{}state} \PY{o}{=} \PY{p}{[}\PY{l+m+mi}{0}\PY{p}{,}\PY{l+m+mi}{1}\PY{p}{]} \PY{c+c1}{\PYZsh{}\PYZsh{} notice we give it in the matrix form, as a list}
\PY{n}{qc}\PY{o}{.}\PY{n}{initialize}\PY{p}{(}\PY{n}{initial\PYZus{}state}\PY{p}{,} \PY{l+m+mi}{0}\PY{p}{)} \PY{c+c1}{\PYZsh{}\PYZsh{} what we want the initial state to be, which qubit}
\PY{n}{qc}\PY{o}{.}\PY{n}{draw}\PY{p}{(}\PY{p}{)}
\end{Verbatim}
\end{tcolorbox}
 
            
\prompt{Out}{outcolor}{28}{}
    
    \begin{center}
    \adjustimage{max size={0.9\linewidth}{0.9\paperheight}}{output_59_0.pdf}
    \end{center}
    { \hspace*{\fill} \\}
    

    We can them use one of Qiksit simulators to view the resulting state of
the qubit:

    \begin{tcolorbox}[breakable, size=fbox, boxrule=1pt, pad at break*=1mm,colback=cellbackground, colframe=cellborder]
\prompt{In}{incolor}{29}{\boxspacing}
\begin{Verbatim}[commandchars=\\\{\}]
\PY{n}{sim} \PY{o}{=} \PY{n}{Aer}\PY{o}{.}\PY{n}{get\PYZus{}backend}\PY{p}{(}\PY{l+s+s1}{\PYZsq{}}\PY{l+s+s1}{aer\PYZus{}simulator}\PY{l+s+s1}{\PYZsq{}}\PY{p}{)}

\PY{n}{qc} \PY{o}{=} \PY{n}{QuantumCircuit}\PY{p}{(}\PY{l+m+mi}{1}\PY{p}{)}\PY{c+c1}{\PYZsh{} Create a quantum circuit with one qubit}
\PY{n}{initial\PYZus{}state} \PY{o}{=} \PY{p}{[}\PY{l+m+mi}{0}\PY{p}{,}\PY{l+m+mi}{1}\PY{p}{]} \PY{c+c1}{\PYZsh{}\PYZsh{} notice we give it in the matrix form, as a list}
\PY{n}{qc}\PY{o}{.}\PY{n}{initialize}\PY{p}{(}\PY{n}{initial\PYZus{}state}\PY{p}{,} \PY{l+m+mi}{0}\PY{p}{)} \PY{c+c1}{\PYZsh{}\PYZsh{} what we want the initial state to be, which qubit}
\PY{n}{qc}\PY{o}{.}\PY{n}{save\PYZus{}statevector}\PY{p}{(}\PY{p}{)} \PY{c+c1}{\PYZsh{}\PYZsh{} tell sim to save statevector, but only possible with sim, obvioulsy}
\PY{n}{result} \PY{o}{=} \PY{n}{sim}\PY{o}{.}\PY{n}{run}\PY{p}{(}\PY{n}{qc}\PY{p}{)}\PY{o}{.}\PY{n}{result}\PY{p}{(}\PY{p}{)}  \PY{c+c1}{\PYZsh{}\PYZsh{}.result gets the result of measurement}

\PY{n}{out\PYZus{}state} \PY{o}{=} \PY{n}{result}\PY{o}{.}\PY{n}{get\PYZus{}statevector}\PY{p}{(}\PY{p}{)}
\PY{n+nb}{print}\PY{p}{(}\PY{n}{out\PYZus{}state}\PY{p}{)} \PY{c+c1}{\PYZsh{}\PYZsh{} which should be the state we gave initially}
\end{Verbatim}
\end{tcolorbox}

    \begin{Verbatim}[commandchars=\\\{\}]
[0.+0.j 1.+0.j]
    \end{Verbatim}

    \begin{tcolorbox}[breakable, size=fbox, boxrule=1pt, pad at break*=1mm,colback=cellbackground, colframe=cellborder]
\prompt{In}{incolor}{30}{\boxspacing}
\begin{Verbatim}[commandchars=\\\{\}]
\PY{n}{qc}\PY{o}{.}\PY{n}{measure\PYZus{}all}\PY{p}{(}\PY{p}{)}
\PY{n}{qc}\PY{o}{.}\PY{n}{draw}\PY{p}{(}\PY{p}{)}
\end{Verbatim}
\end{tcolorbox}
 
            
\prompt{Out}{outcolor}{30}{}
    
    \begin{center}
    \adjustimage{max size={0.9\linewidth}{0.9\paperheight}}{output_62_0.pdf}
    \end{center}
    { \hspace*{\fill} \\}
    

    To see which state we measured, run simulation and
\texttt{get\_counts()}

    \begin{tcolorbox}[breakable, size=fbox, boxrule=1pt, pad at break*=1mm,colback=cellbackground, colframe=cellborder]
\prompt{In}{incolor}{31}{\boxspacing}
\begin{Verbatim}[commandchars=\\\{\}]
\PY{n}{counts} \PY{o}{=} \PY{n}{result}\PY{o}{.}\PY{n}{get\PYZus{}counts}\PY{p}{(}\PY{p}{)}
\PY{n}{plot\PYZus{}histogram}\PY{p}{(}\PY{n}{counts}\PY{p}{)}
\end{Verbatim}
\end{tcolorbox}
 
            
\prompt{Out}{outcolor}{31}{}
    
    \begin{center}
    \adjustimage{max size={0.9\linewidth}{0.9\paperheight}}{output_64_0.pdf}
    \end{center}
    { \hspace*{\fill} \\}
    

    What if instead we tried the same thing with \(|q_o\rangle\)?

    \begin{tcolorbox}[breakable, size=fbox, boxrule=1pt, pad at break*=1mm,colback=cellbackground, colframe=cellborder]
\prompt{In}{incolor}{32}{\boxspacing}
\begin{Verbatim}[commandchars=\\\{\}]
\PY{n}{qc} \PY{o}{=} \PY{n}{QuantumCircuit}\PY{p}{(}\PY{l+m+mi}{1}\PY{p}{)}
\PY{n}{initial\PYZus{}state} \PY{o}{=} \PY{p}{[}\PY{l+m+mi}{1}\PY{o}{/}\PY{n}{sqrt}\PY{p}{(}\PY{l+m+mi}{2}\PY{p}{)}\PY{p}{,} \PY{l+m+mi}{1}\PY{n}{j}\PY{o}{/}\PY{n}{sqrt}\PY{p}{(}\PY{l+m+mi}{2}\PY{p}{)}\PY{p}{]}
\PY{n}{qc}\PY{o}{.}\PY{n}{initialize}\PY{p}{(}\PY{n}{initial\PYZus{}state}\PY{p}{,} \PY{l+m+mi}{0}\PY{p}{)}
\PY{n}{qc}\PY{o}{.}\PY{n}{save\PYZus{}statevector}\PY{p}{(}\PY{p}{)}
\PY{n}{state} \PY{o}{=} \PY{n}{sim}\PY{o}{.}\PY{n}{run}\PY{p}{(}\PY{n}{qc}\PY{p}{)}\PY{o}{.}\PY{n}{result}\PY{p}{(}\PY{p}{)}\PY{o}{.}\PY{n}{get\PYZus{}statevector}\PY{p}{(}\PY{p}{)} \PY{c+c1}{\PYZsh{}\PYZsh{} which is the final statevector?}
\PY{n+nb}{print}\PY{p}{(}\PY{n}{state}\PY{p}{)}
\end{Verbatim}
\end{tcolorbox}

    \begin{Verbatim}[commandchars=\\\{\}]
[0.70710678+0.j         0.        +0.70710678j]
    \end{Verbatim}

    But what if we make a measurement?

    \begin{tcolorbox}[breakable, size=fbox, boxrule=1pt, pad at break*=1mm,colback=cellbackground, colframe=cellborder]
\prompt{In}{incolor}{33}{\boxspacing}
\begin{Verbatim}[commandchars=\\\{\}]
\PY{n}{results} \PY{o}{=} \PY{n}{sim}\PY{o}{.}\PY{n}{run}\PY{p}{(}\PY{n}{qc}\PY{p}{)}\PY{o}{.}\PY{n}{result}\PY{p}{(}\PY{p}{)}\PY{o}{.}\PY{n}{get\PYZus{}counts}\PY{p}{(}\PY{p}{)}
\PY{n}{plot\PYZus{}histogram}\PY{p}{(}\PY{n}{results}\PY{p}{)}
\end{Verbatim}
\end{tcolorbox}
 
            
\prompt{Out}{outcolor}{33}{}
    
    \begin{center}
    \adjustimage{max size={0.9\linewidth}{0.9\paperheight}}{output_68_0.pdf}
    \end{center}
    { \hspace*{\fill} \\}
    

    \hypertarget{the-rules-of-measurement}{%
\subsubsection{The Rules of
Measurement}\label{the-rules-of-measurement}}

    Probability of measuring state \(|\psi\rangle\) into state \(|x\rangle\)

\[
p(|x\rangle) = |\langle x |\psi \rangle|^2 
\]

where the inner product of

\[
\langle a| = 
\begin{bmatrix}
a_1^* & a_2^* & \cdots & a_n^* 
\end{bmatrix}
,
\ \ \ 
|b\rangle  = 
\begin{bmatrix}
b_1 \\ b_2 \\ \cdots \\ b_n 
\end{bmatrix}
\]

is given by

\[
\langle a| b\rangle = 
a_1^* b_1 + a_2^* b_2 + \cdots + a_n^* b_n 
\]

\textbf{In Qiskit, due to normalization, if one tries to initialize a
vector that is not orthonormal, it will give us an error}

    \begin{tcolorbox}[breakable, size=fbox, boxrule=1pt, pad at break*=1mm,colback=cellbackground, colframe=cellborder]
\prompt{In}{incolor}{34}{\boxspacing}
\begin{Verbatim}[commandchars=\\\{\}]
\PY{n}{vector} \PY{o}{=} \PY{p}{[}\PY{l+m+mi}{1}\PY{p}{,}\PY{l+m+mi}{1}\PY{p}{]}
\PY{n}{qc}\PY{o}{.}\PY{n}{initialize}\PY{p}{(}\PY{n}{vector}\PY{p}{,} \PY{l+m+mi}{0}\PY{p}{)}
\end{Verbatim}
\end{tcolorbox}

    \begin{Verbatim}[commandchars=\\\{\}, frame=single, framerule=2mm, rulecolor=\color{outerrorbackground}]
\textcolor{ansi-red}{---------------------------------------------------------------------------}
\textcolor{ansi-red}{QiskitError}                               Traceback (most recent call last)
\textcolor{ansi-green}{<ipython-input-34-ddc73828b990>} in \textcolor{ansi-cyan}{<module>}
\textcolor{ansi-green-intense}{\textbf{      1}} vector \textcolor{ansi-blue}{=} \textcolor{ansi-blue}{[}\textcolor{ansi-cyan}{1}\textcolor{ansi-blue}{,}\textcolor{ansi-cyan}{1}\textcolor{ansi-blue}{]}
\textcolor{ansi-green}{----> 2}\textcolor{ansi-red}{ }qc\textcolor{ansi-blue}{.}initialize\textcolor{ansi-blue}{(}vector\textcolor{ansi-blue}{,} \textcolor{ansi-cyan}{0}\textcolor{ansi-blue}{)}

\textcolor{ansi-green}{\textasciitilde{}/opt/anaconda3/lib/python3.8/site-packages/qiskit/extensions/quantum\_initializer/initializer.py} in \textcolor{ansi-cyan}{initialize}\textcolor{ansi-blue}{(self, params, qubits)}
\textcolor{ansi-green-intense}{\textbf{    457}} 
\textcolor{ansi-green-intense}{\textbf{    458}}     num\_qubits \textcolor{ansi-blue}{=} \textcolor{ansi-green}{None} \textcolor{ansi-green}{if} \textcolor{ansi-green}{not} isinstance\textcolor{ansi-blue}{(}params\textcolor{ansi-blue}{,} int\textcolor{ansi-blue}{)} \textcolor{ansi-green}{else} len\textcolor{ansi-blue}{(}qubits\textcolor{ansi-blue}{)}
\textcolor{ansi-green}{--> 459}\textcolor{ansi-red}{     }\textcolor{ansi-green}{return} self\textcolor{ansi-blue}{.}append\textcolor{ansi-blue}{(}Initialize\textcolor{ansi-blue}{(}params\textcolor{ansi-blue}{,} num\_qubits\textcolor{ansi-blue}{)}\textcolor{ansi-blue}{,} qubits\textcolor{ansi-blue}{)}
\textcolor{ansi-green-intense}{\textbf{    460}} 
\textcolor{ansi-green-intense}{\textbf{    461}} 

\textcolor{ansi-green}{\textasciitilde{}/opt/anaconda3/lib/python3.8/site-packages/qiskit/extensions/quantum\_initializer/initializer.py} in \textcolor{ansi-cyan}{\_\_init\_\_}\textcolor{ansi-blue}{(self, params, num\_qubits)}
\textcolor{ansi-green-intense}{\textbf{     90}}             \textcolor{ansi-red}{\# Check if probabilities (amplitudes squared) sum to 1}
\textcolor{ansi-green-intense}{\textbf{     91}}             \textcolor{ansi-green}{if} \textcolor{ansi-green}{not} math\textcolor{ansi-blue}{.}isclose\textcolor{ansi-blue}{(}sum\textcolor{ansi-blue}{(}np\textcolor{ansi-blue}{.}absolute\textcolor{ansi-blue}{(}params\textcolor{ansi-blue}{)} \textcolor{ansi-blue}{**} \textcolor{ansi-cyan}{2}\textcolor{ansi-blue}{)}\textcolor{ansi-blue}{,} \textcolor{ansi-cyan}{1.0}\textcolor{ansi-blue}{,} abs\_tol\textcolor{ansi-blue}{=}\_EPS\textcolor{ansi-blue}{)}\textcolor{ansi-blue}{:}
\textcolor{ansi-green}{---> 92}\textcolor{ansi-red}{                 }\textcolor{ansi-green}{raise} QiskitError\textcolor{ansi-blue}{(}\textcolor{ansi-blue}{"Sum of amplitudes-squared does not equal one."}\textcolor{ansi-blue}{)}
\textcolor{ansi-green-intense}{\textbf{     93}} 
\textcolor{ansi-green-intense}{\textbf{     94}}             num\_qubits \textcolor{ansi-blue}{=} int\textcolor{ansi-blue}{(}num\_qubits\textcolor{ansi-blue}{)}

\textcolor{ansi-red}{QiskitError}: 'Sum of amplitudes-squared does not equal one.'
    \end{Verbatim}

    Quick exercises: 1. Create a state vector that will give a \(1/3\)
probability of measuring \(|0\rangle\) 2. Create a different state
vector that will give the same measurement probabilities. 3. Verify that
the probability of measuring \(|1\rangle\) for these states is \(2/3\).

    \begin{tcolorbox}[breakable, size=fbox, boxrule=1pt, pad at break*=1mm,colback=cellbackground, colframe=cellborder]
\prompt{In}{incolor}{35}{\boxspacing}
\begin{Verbatim}[commandchars=\\\{\}]
\PY{n}{qc} \PY{o}{=} \PY{n}{QuantumCircuit}\PY{p}{(}\PY{l+m+mi}{1}\PY{p}{)}
\PY{n}{initial\PYZus{}vector\PYZus{}1} \PY{o}{=} \PY{p}{[}\PY{l+m+mi}{1}\PY{o}{/}\PY{n}{sqrt}\PY{p}{(}\PY{l+m+mi}{3}\PY{p}{)}\PY{p}{,} \PY{n}{sqrt}\PY{p}{(}\PY{l+m+mi}{2}\PY{p}{)}\PY{o}{/}\PY{n}{sqrt}\PY{p}{(}\PY{l+m+mi}{3}\PY{p}{)}\PY{p}{]}
\PY{n}{qc}\PY{o}{.}\PY{n}{initialize}\PY{p}{(}\PY{n}{initial\PYZus{}vector\PYZus{}1}\PY{p}{,} \PY{l+m+mi}{0}\PY{p}{)}
\PY{n}{qc}\PY{o}{.}\PY{n}{save\PYZus{}statevector}\PY{p}{(}\PY{p}{)} \PY{c+c1}{\PYZsh{}\PYZsh{} We need to save statevector after initializing it, apparently}
\PY{n}{results} \PY{o}{=} \PY{n}{sim}\PY{o}{.}\PY{n}{run}\PY{p}{(}\PY{n}{qc}\PY{p}{)}\PY{o}{.}\PY{n}{result}\PY{p}{(}\PY{p}{)}\PY{o}{.}\PY{n}{get\PYZus{}counts}\PY{p}{(}\PY{p}{)}
\PY{n}{plot\PYZus{}histogram}\PY{p}{(}\PY{n}{results}\PY{p}{)}
\end{Verbatim}
\end{tcolorbox}
 
            
\prompt{Out}{outcolor}{35}{}
    
    \begin{center}
    \adjustimage{max size={0.9\linewidth}{0.9\paperheight}}{output_73_0.pdf}
    \end{center}
    { \hspace*{\fill} \\}
    

    \begin{tcolorbox}[breakable, size=fbox, boxrule=1pt, pad at break*=1mm,colback=cellbackground, colframe=cellborder]
\prompt{In}{incolor}{36}{\boxspacing}
\begin{Verbatim}[commandchars=\\\{\}]
\PY{n}{qc} \PY{o}{=} \PY{n}{QuantumCircuit}\PY{p}{(}\PY{l+m+mi}{1}\PY{p}{)}
\PY{n}{initial\PYZus{}vector\PYZus{}2} \PY{o}{=} \PY{p}{[}\PY{l+m+mi}{1}\PY{n}{j}\PY{o}{/}\PY{n}{sqrt}\PY{p}{(}\PY{l+m+mi}{3}\PY{p}{)}\PY{p}{,} \PY{o}{\PYZhy{}}\PY{n}{sqrt}\PY{p}{(}\PY{l+m+mi}{2}\PY{p}{)}\PY{o}{/}\PY{n}{sqrt}\PY{p}{(}\PY{l+m+mi}{3}\PY{p}{)}\PY{p}{]}
\PY{n}{qc}\PY{o}{.}\PY{n}{initialize}\PY{p}{(}\PY{n}{initial\PYZus{}vector\PYZus{}2}\PY{p}{,} \PY{l+m+mi}{0}\PY{p}{)}
\PY{c+c1}{\PYZsh{}qc.measure\PYZus{}all()}
\PY{n}{qc}\PY{o}{.}\PY{n}{save\PYZus{}statevector}\PY{p}{(}\PY{p}{)} \PY{c+c1}{\PYZsh{}\PYZsh{} We need to save statevector after initializing it, apparently}
\PY{n}{results} \PY{o}{=} \PY{n}{sim}\PY{o}{.}\PY{n}{run}\PY{p}{(}\PY{n}{qc}\PY{p}{)}\PY{o}{.}\PY{n}{result}\PY{p}{(}\PY{p}{)}\PY{o}{.}\PY{n}{get\PYZus{}counts}\PY{p}{(}\PY{p}{)}
\PY{n}{plot\PYZus{}histogram}\PY{p}{(}\PY{n}{results}\PY{p}{)}
\end{Verbatim}
\end{tcolorbox}
 
            
\prompt{Out}{outcolor}{36}{}
    
    \begin{center}
    \adjustimage{max size={0.9\linewidth}{0.9\paperheight}}{output_74_0.pdf}
    \end{center}
    { \hspace*{\fill} \\}
    

    \#2 And obviously, notice we can also measure in a different basis

\#3 \emph{Global phases}, typically not physically relevant:

\(\gamma\) such that \(|\gamma|=1\)

which are different from \emph{relative phases}.

    \#4 Once measured, we know for certain which state the qubit is

    \begin{tcolorbox}[breakable, size=fbox, boxrule=1pt, pad at break*=1mm,colback=cellbackground, colframe=cellborder]
\prompt{In}{incolor}{37}{\boxspacing}
\begin{Verbatim}[commandchars=\\\{\}]
\PY{n}{qc} \PY{o}{=} \PY{n}{QuantumCircuit}\PY{p}{(}\PY{l+m+mi}{1}\PY{p}{)}
\PY{n}{initial\PYZus{}state} \PY{o}{=} \PY{p}{[}\PY{l+m+mi}{1}\PY{o}{/}\PY{n}{sqrt}\PY{p}{(}\PY{l+m+mi}{2}\PY{p}{)}\PY{p}{,} \PY{l+m+mi}{1}\PY{n}{j}\PY{o}{/}\PY{n}{sqrt}\PY{p}{(}\PY{l+m+mi}{2}\PY{p}{)}\PY{p}{]}
\PY{n}{qc}\PY{o}{.}\PY{n}{initialize}\PY{p}{(}\PY{n}{initial\PYZus{}state}\PY{p}{,} \PY{l+m+mi}{0}\PY{p}{)}
\PY{n}{qc}\PY{o}{.}\PY{n}{draw}\PY{p}{(}\PY{p}{)}
\end{Verbatim}
\end{tcolorbox}
 
            
\prompt{Out}{outcolor}{37}{}
    
    \begin{center}
    \adjustimage{max size={0.9\linewidth}{0.9\paperheight}}{output_77_0.pdf}
    \end{center}
    { \hspace*{\fill} \\}
    

    But after the measurement

    \begin{tcolorbox}[breakable, size=fbox, boxrule=1pt, pad at break*=1mm,colback=cellbackground, colframe=cellborder]
\prompt{In}{incolor}{38}{\boxspacing}
\begin{Verbatim}[commandchars=\\\{\}]
\PY{n}{qc} \PY{o}{=} \PY{n}{QuantumCircuit}\PY{p}{(}\PY{l+m+mi}{1}\PY{p}{)}
\PY{n}{initial\PYZus{}state} \PY{o}{=} \PY{p}{[}\PY{l+m+mi}{1}\PY{o}{/}\PY{n}{sqrt}\PY{p}{(}\PY{l+m+mi}{2}\PY{p}{)}\PY{p}{,} \PY{l+m+mi}{1}\PY{n}{j}\PY{o}{/}\PY{n}{sqrt}\PY{p}{(}\PY{l+m+mi}{2}\PY{p}{)}\PY{p}{]}
\PY{n}{qc}\PY{o}{.}\PY{n}{initialize}\PY{p}{(}\PY{n}{initial\PYZus{}state}\PY{p}{,} \PY{l+m+mi}{0}\PY{p}{)}
\PY{n}{qc}\PY{o}{.}\PY{n}{measure\PYZus{}all}\PY{p}{(}\PY{p}{)}
\PY{n}{qc}\PY{o}{.}\PY{n}{save\PYZus{}statevector}\PY{p}{(}\PY{p}{)}
\PY{n}{qc}\PY{o}{.}\PY{n}{draw}\PY{p}{(}\PY{p}{)}
\end{Verbatim}
\end{tcolorbox}
 
            
\prompt{Out}{outcolor}{38}{}
    
    \begin{center}
    \adjustimage{max size={0.9\linewidth}{0.9\paperheight}}{output_79_0.pdf}
    \end{center}
    { \hspace*{\fill} \\}
    

    \begin{tcolorbox}[breakable, size=fbox, boxrule=1pt, pad at break*=1mm,colback=cellbackground, colframe=cellborder]
\prompt{In}{incolor}{39}{\boxspacing}
\begin{Verbatim}[commandchars=\\\{\}]
\PY{n}{state} \PY{o}{=} \PY{n}{sim}\PY{o}{.}\PY{n}{run}\PY{p}{(}\PY{n}{qc}\PY{p}{)}\PY{o}{.}\PY{n}{result}\PY{p}{(}\PY{p}{)}\PY{o}{.}\PY{n}{get\PYZus{}statevector}\PY{p}{(}\PY{p}{)}
\PY{n+nb}{print}\PY{p}{(}\PY{n}{state}\PY{p}{)}
\end{Verbatim}
\end{tcolorbox}

    \begin{Verbatim}[commandchars=\\\{\}]
[1.+0.j 0.+0.j]
    \end{Verbatim}

    We can see that writing down a qubit's state requires keeping track of
two complex numbers, but when using a real quantum computer we will only
ever receive a yes-or-no (0 or 1) answer for each qubit. The output of a
10-qubit quantum computer will look like this:

\(0110111110\)

Just 10 bits, no superposition or complex amplitudes. When using a real
quantum computer, we cannot see the states of our qubits
mid-computation, as this would destroy them! This behaviour is not ideal
for learning, so Qiskit provides different quantum simulators: By
default, the aer\_simulator mimics the execution of a real quantum
computer, but will also allow you to peek at quantum states before
measurement if we include certain instructions in our circuit. For
example, here we have included the instruction
\texttt{.save\_statevector()}, which means we can use
\texttt{.get\_statevector()} on the result of the simulation.

    \hypertarget{the-bloch-sphere}{%
\subsubsection{The Bloch Sphere}\label{the-bloch-sphere}}

    The Qubit is described by

\[|q\rangle = \alpha|0\rangle + \beta|1\rangle, \ \ \alpha, \beta \  \mathrm{in} \ \mathbb{C} \]

but since we cannot observe global phases, we can remove one of the
degrees of freedom and rewrite this as

\[|q\rangle = \alpha|0\rangle + e^{i\phi}\beta|1\rangle, \ \ \alpha, \beta, \phi \  \mathrm{in} \ \mathbb{R} \]

and given the normalization \(|\alpha|^2 + |\beta|^2 = 1\), we can use
the trigonometric identity \(\sin^2 x + \cos^2 x = 1\) to describe
\(\alpha, \beta\) in \(\mathbb{R}\) in terms of a single angle
\(\theta\),

\[ \alpha = \cos\frac{\theta}{2}, \ \ \beta = \sin\frac{\theta}{2},\]

and describe the qubit using two angles \(\phi\) and \(\theta\),

\[|q\rangle = \cos\frac{\theta}{2}|0\rangle + e^{i\phi}\sin\frac{\theta}{2}|1\rangle, \ \ \theta, \phi \  \mathrm{in} \ \mathbb{R}\]

    This means we can represent these states in a Bloch sphere:

    \begin{tcolorbox}[breakable, size=fbox, boxrule=1pt, pad at break*=1mm,colback=cellbackground, colframe=cellborder]
\prompt{In}{incolor}{40}{\boxspacing}
\begin{Verbatim}[commandchars=\\\{\}]
\PY{k+kn}{from} \PY{n+nn}{qiskit\PYZus{}textbook}\PY{n+nn}{.}\PY{n+nn}{widgets} \PY{k+kn}{import} \PY{n}{plot\PYZus{}bloch\PYZus{}vector\PYZus{}spherical}
\PY{n}{coords} \PY{o}{=} \PY{p}{[}\PY{n}{pi}\PY{o}{/}\PY{l+m+mi}{2}\PY{p}{,} \PY{l+m+mi}{0}\PY{p}{,} \PY{l+m+mi}{1}\PY{p}{]} \PY{c+c1}{\PYZsh{}\PYZsh{} theta, phi, radius}
\PY{n}{plot\PYZus{}bloch\PYZus{}vector\PYZus{}spherical}\PY{p}{(}\PY{n}{coords}\PY{p}{)}
\end{Verbatim}
\end{tcolorbox}
 
            
\prompt{Out}{outcolor}{40}{}
    
    \begin{center}
    \adjustimage{max size={0.9\linewidth}{0.9\paperheight}}{output_85_0.pdf}
    \end{center}
    { \hspace*{\fill} \\}
    

    \textbf{Do not confuse the \emph{Bloch vector} with the
\emph{statevector}. The Bloch vector is a visualisation tool that maps
the 2D, complex statevector onto real, 3D space.}

    \begin{tcolorbox}[breakable, size=fbox, boxrule=1pt, pad at break*=1mm,colback=cellbackground, colframe=cellborder]
\prompt{In}{incolor}{41}{\boxspacing}
\begin{Verbatim}[commandchars=\\\{\}]
\PY{c+c1}{\PYZsh{} 1: state 0 \PYZhy{}\PYZgt{} theta = 0, phi = 0}

\PY{n}{coords} \PY{o}{=} \PY{p}{[}\PY{l+m+mi}{0}\PY{p}{,} \PY{l+m+mi}{0}\PY{p}{,} \PY{l+m+mi}{1}\PY{p}{]}
\PY{n}{plot\PYZus{}bloch\PYZus{}vector\PYZus{}spherical}\PY{p}{(}\PY{n}{coords}\PY{p}{)}
\end{Verbatim}
\end{tcolorbox}
 
            
\prompt{Out}{outcolor}{41}{}
    
    \begin{center}
    \adjustimage{max size={0.9\linewidth}{0.9\paperheight}}{output_87_0.pdf}
    \end{center}
    { \hspace*{\fill} \\}
    

    \begin{tcolorbox}[breakable, size=fbox, boxrule=1pt, pad at break*=1mm,colback=cellbackground, colframe=cellborder]
\prompt{In}{incolor}{42}{\boxspacing}
\begin{Verbatim}[commandchars=\\\{\}]
\PY{c+c1}{\PYZsh{} 2: state 1 \PYZhy{}\PYZgt{} theta = pi, phi = 0}
\PY{n}{coords} \PY{o}{=} \PY{p}{[}\PY{n}{pi}\PY{p}{,} \PY{l+m+mi}{0}\PY{p}{,} \PY{l+m+mi}{1}\PY{p}{]}
\PY{n}{plot\PYZus{}bloch\PYZus{}vector\PYZus{}spherical}\PY{p}{(}\PY{n}{coords}\PY{p}{)}
\end{Verbatim}
\end{tcolorbox}
 
            
\prompt{Out}{outcolor}{42}{}
    
    \begin{center}
    \adjustimage{max size={0.9\linewidth}{0.9\paperheight}}{output_88_0.pdf}
    \end{center}
    { \hspace*{\fill} \\}
    

    \begin{tcolorbox}[breakable, size=fbox, boxrule=1pt, pad at break*=1mm,colback=cellbackground, colframe=cellborder]
\prompt{In}{incolor}{43}{\boxspacing}
\begin{Verbatim}[commandchars=\\\{\}]
\PY{c+c1}{\PYZsh{} 3: state (1/sqrt(2))(0+1) \PYZhy{}\PYZgt{} theta = pi/2, phi = 0}
\PY{n}{coords} \PY{o}{=} \PY{p}{[}\PY{n}{pi}\PY{o}{/}\PY{l+m+mi}{2}\PY{p}{,} \PY{l+m+mi}{0}\PY{p}{,} \PY{l+m+mi}{1}\PY{p}{]}
\PY{n}{plot\PYZus{}bloch\PYZus{}vector\PYZus{}spherical}\PY{p}{(}\PY{n}{coords}\PY{p}{)}
\end{Verbatim}
\end{tcolorbox}
 
            
\prompt{Out}{outcolor}{43}{}
    
    \begin{center}
    \adjustimage{max size={0.9\linewidth}{0.9\paperheight}}{output_89_0.pdf}
    \end{center}
    { \hspace*{\fill} \\}
    

    \begin{tcolorbox}[breakable, size=fbox, boxrule=1pt, pad at break*=1mm,colback=cellbackground, colframe=cellborder]
\prompt{In}{incolor}{44}{\boxspacing}
\begin{Verbatim}[commandchars=\\\{\}]
\PY{c+c1}{\PYZsh{} 4: tate (1/sqrt(2))(0\PYZhy{}i*1) \PYZhy{}\PYZgt{} theta = pi/2, phi = 3pi/2}
\PY{n}{coords} \PY{o}{=} \PY{p}{[}\PY{n}{pi}\PY{o}{/}\PY{l+m+mi}{2}\PY{p}{,} \PY{l+m+mi}{3}\PY{o}{*}\PY{n}{pi}\PY{o}{/}\PY{l+m+mi}{2}\PY{p}{,} \PY{l+m+mi}{1}\PY{p}{]}
\PY{n}{plot\PYZus{}bloch\PYZus{}vector\PYZus{}spherical}\PY{p}{(}\PY{n}{coords}\PY{p}{)}
\end{Verbatim}
\end{tcolorbox}
 
            
\prompt{Out}{outcolor}{44}{}
    
    \begin{center}
    \adjustimage{max size={0.9\linewidth}{0.9\paperheight}}{output_90_0.pdf}
    \end{center}
    { \hspace*{\fill} \\}
    

    \begin{tcolorbox}[breakable, size=fbox, boxrule=1pt, pad at break*=1mm,colback=cellbackground, colframe=cellborder]
\prompt{In}{incolor}{45}{\boxspacing}
\begin{Verbatim}[commandchars=\\\{\}]
\PY{c+c1}{\PYZsh{} 5: tate (1/sqrt(2))(0\PYZhy{}i*1) \PYZhy{}\PYZgt{} theta = pi/2, phi = 3pi/2}
\PY{n}{coords} \PY{o}{=} \PY{p}{[}\PY{n}{pi}\PY{o}{/}\PY{l+m+mi}{2}\PY{p}{,} \PY{l+m+mi}{3}\PY{o}{*}\PY{n}{pi}\PY{o}{/}\PY{l+m+mi}{2}\PY{p}{,} \PY{l+m+mi}{1}\PY{p}{]}
\PY{n}{plot\PYZus{}bloch\PYZus{}vector\PYZus{}spherical}\PY{p}{(}\PY{n}{coords}\PY{p}{)}
\end{Verbatim}
\end{tcolorbox}
 
            
\prompt{Out}{outcolor}{45}{}
    
    \begin{center}
    \adjustimage{max size={0.9\linewidth}{0.9\paperheight}}{output_91_0.pdf}
    \end{center}
    { \hspace*{\fill} \\}
    

    \begin{tcolorbox}[breakable, size=fbox, boxrule=1pt, pad at break*=1mm,colback=cellbackground, colframe=cellborder]
\prompt{In}{incolor}{ }{\boxspacing}
\begin{Verbatim}[commandchars=\\\{\}]

\end{Verbatim}
\end{tcolorbox}

    \begin{tcolorbox}[breakable, size=fbox, boxrule=1pt, pad at break*=1mm,colback=cellbackground, colframe=cellborder]
\prompt{In}{incolor}{ }{\boxspacing}
\begin{Verbatim}[commandchars=\\\{\}]

\end{Verbatim}
\end{tcolorbox}

    \begin{tcolorbox}[breakable, size=fbox, boxrule=1pt, pad at break*=1mm,colback=cellbackground, colframe=cellborder]
\prompt{In}{incolor}{46}{\boxspacing}
\begin{Verbatim}[commandchars=\\\{\}]
\PY{k+kn}{from} \PY{n+nn}{qiskit\PYZus{}textbook}\PY{n+nn}{.}\PY{n+nn}{widgets} \PY{k+kn}{import} \PY{n}{bloch\PYZus{}calc}
\PY{n}{bloch\PYZus{}calc}\PY{p}{(}\PY{p}{)}
\end{Verbatim}
\end{tcolorbox}

    
    \begin{Verbatim}[commandchars=\\\{\}]
VBox(children=(Label(value='Define a qubit state using \$\textbackslash{}\textbackslash{}theta\$ and \$\textbackslash{}\textbackslash{}phi\$:'), Text(value='', placeholder='T…
    \end{Verbatim}

    
    
    \begin{Verbatim}[commandchars=\\\{\}]
HTML(value='<pre></pre>')
    \end{Verbatim}

    
    
    \begin{Verbatim}[commandchars=\\\{\}]
Image(value=b'\textbackslash{}x89PNG\textbackslash{}r\textbackslash{}n\textbackslash{}x1a\textbackslash{}n\textbackslash{}x00\textbackslash{}x00\textbackslash{}x00\textbackslash{}rIHDR\textbackslash{}x00\textbackslash{}x00\textbackslash{}x01h\textbackslash{}x00\textbackslash{}x00\textbackslash{}x01h\textbackslash{}x08\textbackslash{}x06\textbackslash{}x00\textbackslash{}x00\textbackslash{}x00z\textbackslash{}xe5a\textbackslash{}xd5\textbackslash{}x00\textbackslash{}…
    \end{Verbatim}

    
    \begin{tcolorbox}[breakable, size=fbox, boxrule=1pt, pad at break*=1mm,colback=cellbackground, colframe=cellborder]
\prompt{In}{incolor}{ }{\boxspacing}
\begin{Verbatim}[commandchars=\\\{\}]

\end{Verbatim}
\end{tcolorbox}

    \hypertarget{single-qubit-gates}{%
\subsection{Single Qubit Gates}\label{single-qubit-gates}}

    In this section we will cover gates, the operations that change a qubit
between these states.

    \begin{tcolorbox}[breakable, size=fbox, boxrule=1pt, pad at break*=1mm,colback=cellbackground, colframe=cellborder]
\prompt{In}{incolor}{47}{\boxspacing}
\begin{Verbatim}[commandchars=\\\{\}]
\PY{k+kn}{from} \PY{n+nn}{qiskit} \PY{k+kn}{import} \PY{n}{QuantumCircuit}\PY{p}{,} \PY{n}{Aer}
\PY{k+kn}{from} \PY{n+nn}{math} \PY{k+kn}{import} \PY{n}{pi}\PY{p}{,} \PY{n}{sqrt}
\PY{k+kn}{from} \PY{n+nn}{qiskit}\PY{n+nn}{.}\PY{n+nn}{visualization} \PY{k+kn}{import} \PY{n}{plot\PYZus{}bloch\PYZus{}multivector}\PY{p}{,} \PY{n}{plot\PYZus{}histogram}
\PY{n}{sim} \PY{o}{=} \PY{n}{Aer}\PY{o}{.}\PY{n}{get\PYZus{}backend}\PY{p}{(}\PY{l+s+s1}{\PYZsq{}}\PY{l+s+s1}{aer\PYZus{}simulator}\PY{l+s+s1}{\PYZsq{}}\PY{p}{)}
\end{Verbatim}
\end{tcolorbox}

    \hypertarget{the-pauli-gates}{%
\subsubsection{The Pauli Gates}\label{the-pauli-gates}}

    \hypertarget{the-x-gate}{%
\paragraph{The X-Gate}\label{the-x-gate}}

    Pauli-X matrix

\[
X = 
\begin{bmatrix}
0 & 1 \\
1 & 0
\end{bmatrix}
=
|0\rangle \langle 1| + |1\rangle \langle 0|
\]

To see the effect of the gate, let's apply to \(|0\rangle\) \&
\(|1\rangle\).

\[
X|0\rangle = 
\begin{bmatrix}
0 & 1 \\
1 & 0
\end{bmatrix}
\begin{bmatrix}
1  \\
0
\end{bmatrix}
=
\begin{bmatrix}
0  \\
1
\end{bmatrix}
=
|1\rangle
\]

\(\pi\) rotation around the X-axis

    \begin{tcolorbox}[breakable, size=fbox, boxrule=1pt, pad at break*=1mm,colback=cellbackground, colframe=cellborder]
\prompt{In}{incolor}{48}{\boxspacing}
\begin{Verbatim}[commandchars=\\\{\}]
\PY{n}{qc} \PY{o}{=} \PY{n}{QuantumCircuit}\PY{p}{(}\PY{l+m+mi}{1}\PY{p}{)}
\PY{n}{qc}\PY{o}{.}\PY{n}{x}\PY{p}{(}\PY{l+m+mi}{0}\PY{p}{)}
\PY{n}{qc}\PY{o}{.}\PY{n}{draw}\PY{p}{(}\PY{p}{)}
\end{Verbatim}
\end{tcolorbox}
 
            
\prompt{Out}{outcolor}{48}{}
    
    \begin{center}
    \adjustimage{max size={0.9\linewidth}{0.9\paperheight}}{output_102_0.pdf}
    \end{center}
    { \hspace*{\fill} \\}
    

    \begin{tcolorbox}[breakable, size=fbox, boxrule=1pt, pad at break*=1mm,colback=cellbackground, colframe=cellborder]
\prompt{In}{incolor}{49}{\boxspacing}
\begin{Verbatim}[commandchars=\\\{\}]
\PY{n}{qc}\PY{o}{.}\PY{n}{save\PYZus{}statevector}\PY{p}{(}\PY{p}{)}
\PY{n}{state} \PY{o}{=} \PY{n}{sim}\PY{o}{.}\PY{n}{run}\PY{p}{(}\PY{n}{qc}\PY{p}{)}\PY{o}{.}\PY{n}{result}\PY{p}{(}\PY{p}{)}\PY{o}{.}\PY{n}{get\PYZus{}statevector}\PY{p}{(}\PY{p}{)}
\PY{n}{plot\PYZus{}bloch\PYZus{}multivector}\PY{p}{(}\PY{n}{state}\PY{p}{)}
\end{Verbatim}
\end{tcolorbox}
 
            
\prompt{Out}{outcolor}{49}{}
    
    \begin{center}
    \adjustimage{max size={0.9\linewidth}{0.9\paperheight}}{output_103_0.pdf}
    \end{center}
    { \hspace*{\fill} \\}
    

    \hypertarget{the-y-z-gates}{%
\paragraph{The Y \& Z-Gates}\label{the-y-z-gates}}

    \[
Y = 
\begin{bmatrix}
0 & -i \\
i & 0
\end{bmatrix}
=
-i|0\rangle \langle 1| + i|1\rangle \langle 0|
\]

\[
Z = 
\begin{bmatrix}
1 & 0 \\
0 & -1
\end{bmatrix}
=
|0\rangle \langle 0| - |1\rangle \langle 1|
\]

\(\pi\) rotation around y- and z-axis, respectively.

    \begin{tcolorbox}[breakable, size=fbox, boxrule=1pt, pad at break*=1mm,colback=cellbackground, colframe=cellborder]
\prompt{In}{incolor}{50}{\boxspacing}
\begin{Verbatim}[commandchars=\\\{\}]
\PY{k+kn}{from} \PY{n+nn}{qiskit\PYZus{}textbook}\PY{n+nn}{.}\PY{n+nn}{widgets} \PY{k+kn}{import} \PY{n}{gate\PYZus{}demo}
\PY{n}{gate\PYZus{}demo}\PY{p}{(}\PY{n}{gates}\PY{o}{=}\PY{l+s+s1}{\PYZsq{}}\PY{l+s+s1}{pauli}\PY{l+s+s1}{\PYZsq{}}\PY{p}{)}
\end{Verbatim}
\end{tcolorbox}

    
    \begin{Verbatim}[commandchars=\\\{\}]
HBox(children=(Button(description='X', layout=Layout(height='3em', width='3em'), style=ButtonStyle()), Button(…
    \end{Verbatim}

    
    
    \begin{Verbatim}[commandchars=\\\{\}]
Image(value=b'\textbackslash{}x89PNG\textbackslash{}r\textbackslash{}n\textbackslash{}x1a\textbackslash{}n\textbackslash{}x00\textbackslash{}x00\textbackslash{}x00\textbackslash{}rIHDR\textbackslash{}x00\textbackslash{}x00\textbackslash{}x01 \textbackslash{}x00\textbackslash{}x00\textbackslash{}x01 \textbackslash{}x08\textbackslash{}x06\textbackslash{}x00\textbackslash{}x00\textbackslash{}x00\textbackslash{}x14\textbackslash{}x83\textbackslash{}xae\textbackslash{}x8…
    \end{Verbatim}

    
    \begin{tcolorbox}[breakable, size=fbox, boxrule=1pt, pad at break*=1mm,colback=cellbackground, colframe=cellborder]
\prompt{In}{incolor}{51}{\boxspacing}
\begin{Verbatim}[commandchars=\\\{\}]
\PY{n}{qc}\PY{o}{.}\PY{n}{y}\PY{p}{(}\PY{l+m+mi}{0}\PY{p}{)}
\PY{n}{qc}\PY{o}{.}\PY{n}{z}\PY{p}{(}\PY{l+m+mi}{0}\PY{p}{)} 
\PY{n}{qc}\PY{o}{.}\PY{n}{draw}\PY{p}{(}\PY{p}{)}
\end{Verbatim}
\end{tcolorbox}
 
            
\prompt{Out}{outcolor}{51}{}
    
    \begin{center}
    \adjustimage{max size={0.9\linewidth}{0.9\paperheight}}{output_107_0.pdf}
    \end{center}
    { \hspace*{\fill} \\}
    

    \begin{tcolorbox}[breakable, size=fbox, boxrule=1pt, pad at break*=1mm,colback=cellbackground, colframe=cellborder]
\prompt{In}{incolor}{ }{\boxspacing}
\begin{Verbatim}[commandchars=\\\{\}]

\end{Verbatim}
\end{tcolorbox}

    \hypertarget{digression-the-x-y-z-bases}{%
\subsubsection{Digression: the X, Y \& Z
Bases}\label{digression-the-x-y-z-bases}}

    \(|0\rangle\) and \(|1\rangle\) are the two eigenstatesof the Z-gate. In
fact, the \emph{computational basis} (\(|0\rangle\) and \(|1\rangle\))
is often called \emph{Z-basis}. Another popular basis is the
\emph{X-basis}, the eigenstates of the X-gate:

\[
|+\rangle 
= \frac{1}{\sqrt{2}}(|0\rangle + |1\rangle) 
= \frac{1}{\sqrt{2}}
\begin{bmatrix}
1\\
1
\end{bmatrix}
\]

\[
|-\rangle 
= \frac{1}{\sqrt{2}}(|0\rangle - |1\rangle) 
= \frac{1}{\sqrt{2}}
\begin{bmatrix}
1\\
-1
\end{bmatrix}
\]

    The less common are the eigenstates of the Y-gate,
\(|\circlearrowleft \rangle\) and \(| \circlearrowright \rangle\)

    \textbf{Quick exercises}

\#1 Verify that \(|+\rangle\) and \(|-\rangle\) are in fact eigenstates
of the X-gate.

\#2. What eigenvalues do they have?

\[
\begin{eqnarray}
X |+\rangle &= |+\rangle \\
X |-\rangle &= -|-\rangle
\end{eqnarray}
\]

\#3 Find the eigenstates of the Y-gate, and their co-ordinates on the
Bloch sphere.

\textbf{Eigenvalues}: \[
\begin{bmatrix}
-\lambda & -i \\
i & -\lambda
\end{bmatrix}
\rightarrow \lambda^2 = 1 \rightarrow \lambda = \pm 1
\]

\textbf{Eigenvectors}: \[
\lambda = +1, \alpha=1 \rightarrow 
\begin{bmatrix}
-i\beta \\
i
\end{bmatrix}
=
\begin{bmatrix}
1 \\
\beta
\end{bmatrix}
\rightarrow \beta = i 
\rightarrow |\lambda\rangle_+ = \frac{1}{\sqrt{2}}
\begin{bmatrix}
1 \\
i
\end{bmatrix}
= \frac{1}{\sqrt{2}}(|0\rangle + i|1\rangle)
\]

\[
\lambda = -1, \alpha=1 \rightarrow 
\begin{bmatrix}
-i\beta \\
i
\end{bmatrix}
=
\begin{bmatrix}
-1 \\
-\beta
\end{bmatrix}
\rightarrow \beta = -i 
\rightarrow 
|\lambda\rangle_- = \frac{1}{\sqrt{2}}
\begin{bmatrix}
1 \\
-i
\end{bmatrix} 
= \frac{1}{\sqrt{2}}(|0\rangle - i|1\rangle)
\]

    \begin{tcolorbox}[breakable, size=fbox, boxrule=1pt, pad at break*=1mm,colback=cellbackground, colframe=cellborder]
\prompt{In}{incolor}{52}{\boxspacing}
\begin{Verbatim}[commandchars=\\\{\}]
\PY{k+kn}{import} \PY{n+nn}{numpy} \PY{k}{as} \PY{n+nn}{np}
\PY{k+kn}{from} \PY{n+nn}{numpy}\PY{n+nn}{.}\PY{n+nn}{linalg} \PY{k+kn}{import} \PY{n}{eig}
\PY{n}{a} \PY{o}{=} \PY{n}{np}\PY{o}{.}\PY{n}{array}\PY{p}{(}\PY{p}{[}\PY{p}{[}\PY{l+m+mi}{0}\PY{p}{,} \PY{l+m+mi}{0}\PY{o}{\PYZhy{}}\PY{l+m+mi}{1}\PY{n}{j}\PY{p}{]}\PY{p}{,} 
              \PY{p}{[}\PY{l+m+mi}{0}\PY{o}{+}\PY{l+m+mi}{1}\PY{n}{j}\PY{p}{,} \PY{l+m+mi}{0}\PY{p}{]}\PY{p}{]}\PY{p}{)}
\PY{n+nb}{print}\PY{p}{(}\PY{n}{a}\PY{p}{)}
\PY{n}{w}\PY{p}{,}\PY{n}{v}\PY{o}{=}\PY{n}{eig}\PY{p}{(}\PY{n}{a}\PY{p}{)}

\PY{n+nb}{print}\PY{p}{(}\PY{l+s+s1}{\PYZsq{}}\PY{l+s+s1}{E\PYZhy{}value:}\PY{l+s+s1}{\PYZsq{}}\PY{p}{,} \PY{n}{w}\PY{p}{,} \PY{l+s+s1}{\PYZsq{}}\PY{l+s+se}{\PYZbs{}n}\PY{l+s+s1}{\PYZsq{}}\PY{p}{)}
\PY{n+nb}{print}\PY{p}{(}\PY{l+s+s1}{\PYZsq{}}\PY{l+s+s1}{E\PYZhy{}vector}\PY{l+s+s1}{\PYZsq{}}\PY{p}{,} \PY{n}{v}\PY{p}{)}
\end{Verbatim}
\end{tcolorbox}

    \begin{Verbatim}[commandchars=\\\{\}]
[[0.+0.j 0.-1.j]
 [0.+1.j 0.+0.j]]
E-value: [ 1.+0.j -1.+0.j]

E-vector [[-0.        -0.70710678j  0.70710678+0.j        ]
 [ 0.70710678+0.j          0.        -0.70710678j]]
    \end{Verbatim}

    \textbf{Using only Paulis we cannot move the qubit away from
\(|0\rangle\), \(|1\rangle\) and we cannot achieve superpositions.}

    \begin{tcolorbox}[breakable, size=fbox, boxrule=1pt, pad at break*=1mm,colback=cellbackground, colframe=cellborder]
\prompt{In}{incolor}{ }{\boxspacing}
\begin{Verbatim}[commandchars=\\\{\}]

\end{Verbatim}
\end{tcolorbox}

    \hypertarget{the-hadamard-gate}{%
\subsubsection{The Hadamard Gate}\label{the-hadamard-gate}}

    H-Gate

\[
H= \frac{1}{\sqrt(2)}
\begin{bmatrix}
1 & 1 \\
1 & -1
\end{bmatrix}
\]

Which allows the transformations

\[
H| 0 \rangle = \frac{1}{\sqrt{2}} 
\begin{bmatrix}
1 \\
1
\end{bmatrix} 
= |+\rangle
\]

\[
H| 1 \rangle = \frac{1}{\sqrt{2}} 
\begin{bmatrix}
1 \\
-1
\end{bmatrix} 
= |-\rangle
\]

Which can be thought as a rotation of \(\pi/2\) around the Bloch vector
\([1, 0,1]\) (or around the y-axis), transforming the state of the qubit
between the X and Z bases.

    \begin{tcolorbox}[breakable, size=fbox, boxrule=1pt, pad at break*=1mm,colback=cellbackground, colframe=cellborder]
\prompt{In}{incolor}{53}{\boxspacing}
\begin{Verbatim}[commandchars=\\\{\}]
\PY{k+kn}{from} \PY{n+nn}{qiskit\PYZus{}textbook}\PY{n+nn}{.}\PY{n+nn}{widgets} \PY{k+kn}{import} \PY{n}{gate\PYZus{}demo}
\PY{n}{gate\PYZus{}demo}\PY{p}{(}\PY{n}{gates}\PY{o}{=}\PY{l+s+s1}{\PYZsq{}}\PY{l+s+s1}{pauli+h}\PY{l+s+s1}{\PYZsq{}}\PY{p}{)}
\end{Verbatim}
\end{tcolorbox}

    
    \begin{Verbatim}[commandchars=\\\{\}]
HBox(children=(Button(description='X', layout=Layout(height='3em', width='3em'), style=ButtonStyle()), Button(…
    \end{Verbatim}

    
    
    \begin{Verbatim}[commandchars=\\\{\}]
Image(value=b'\textbackslash{}x89PNG\textbackslash{}r\textbackslash{}n\textbackslash{}x1a\textbackslash{}n\textbackslash{}x00\textbackslash{}x00\textbackslash{}x00\textbackslash{}rIHDR\textbackslash{}x00\textbackslash{}x00\textbackslash{}x01 \textbackslash{}x00\textbackslash{}x00\textbackslash{}x01 \textbackslash{}x08\textbackslash{}x06\textbackslash{}x00\textbackslash{}x00\textbackslash{}x00\textbackslash{}x14\textbackslash{}x83\textbackslash{}xae\textbackslash{}x8…
    \end{Verbatim}

    
    \textbf{Quick exercises:}

\begin{itemize}
\tightlist
\item
  \#1 Write the H-gate as the outer products of vectors
  \(|0\rangle, |1\rangle, |+\rangle, |-\rangle\)
\end{itemize}

We have that:

\[
\begin{eqnarray}
|0\rangle \langle 0| &= 
\begin{bmatrix}
1 & 0 \\
0 & 0
\end{bmatrix}\\
|0\rangle \langle 1| &=
\begin{bmatrix}
0 & 1 \\
0 & 0
\end{bmatrix}\\
|1\rangle \langle 0| &= 
\begin{bmatrix}
0 & 0 \\
1 & 0
\end{bmatrix}\\
|1\rangle \langle 1| &= 
\begin{bmatrix}
0 & 0 \\
0 & 1
\end{bmatrix}
\end{eqnarray}
\]

and

\[
\begin{eqnarray}
|+\rangle \langle +| &= \frac{1}{2}
\begin{bmatrix}
1 & 1 \\
1 & 1
\end{bmatrix}\\
|+\rangle \langle -| &=\frac{1}{2}
\begin{bmatrix}
1 & -1 \\
1 & -1
\end{bmatrix}\\
|-\rangle \langle +| &= \frac{1}{2}
\begin{bmatrix}
1 & 1 \\
-1 & -1
\end{bmatrix}\\
|-\rangle \langle -| &= \frac{1}{2}
\begin{bmatrix}
1 & -1 \\
-1 & 1
\end{bmatrix}
\end{eqnarray}
\]

and therefore,

\[
H = \frac{1}{\sqrt{2}}\left[ |0\rangle \langle 0| 
+ |0\rangle \langle 1| + |1\rangle \langle 0| - |1\rangle \langle 1|\right]
=
\frac{1}{\sqrt{8}}\left[ |+\rangle \langle +| 
+ |+\rangle \langle -| + |-\rangle \langle +| - |-\rangle \langle -|\right]
\]

\begin{itemize}
\tightlist
\item
  \#2 Show that applying the sequence of gates: HZH, to any qubit state
  is equivalent to applying an X-gate.
\end{itemize}

\[
HZH = \frac{1}{2}
\begin{bmatrix}
1 & 1 \\
-1 & -1
\end{bmatrix}
\begin{bmatrix}
1 & 0 \\
0 & -1
\end{bmatrix}
\begin{bmatrix}
1 & 1 \\
-1 & -1
\end{bmatrix} 
= \frac{1}{2}
\begin{bmatrix}
1 & -1 \\
1 & 1
\end{bmatrix}
\begin{bmatrix}
1 & 1 \\
-1 & -1
\end{bmatrix} 
=\frac{1}{2}
\begin{bmatrix}
0 & 2 \\
2 & 0
\end{bmatrix}
= X
\]

\begin{itemize}
\tightlist
\item
  \#3 Find a combination of X, Z and H-gates that is equivalent to a
  Y-gate (ignoring global phase).
\end{itemize}

From the properties of Pauli matrices, \(ZX = iY\)

    \begin{tcolorbox}[breakable, size=fbox, boxrule=1pt, pad at break*=1mm,colback=cellbackground, colframe=cellborder]
\prompt{In}{incolor}{ }{\boxspacing}
\begin{Verbatim}[commandchars=\\\{\}]

\end{Verbatim}
\end{tcolorbox}

    \hypertarget{digression-measuring-in-different-bases}{%
\subsubsection{Digression: Measuring in Different
Bases}\label{digression-measuring-in-different-bases}}

    Since Qiskit only allows measuring in the Z-basis, if we want to make
measurements in a different basis we must create it using Hadamard
gates. To measure on the X-basis, for instance:

    \begin{tcolorbox}[breakable, size=fbox, boxrule=1pt, pad at break*=1mm,colback=cellbackground, colframe=cellborder]
\prompt{In}{incolor}{54}{\boxspacing}
\begin{Verbatim}[commandchars=\\\{\}]
\PY{c+c1}{\PYZsh{}\PYZsh{} Create the X\PYZhy{}measurememnt function}
\PY{k}{def} \PY{n+nf}{x\PYZus{}measurement}\PY{p}{(}\PY{n}{qc}\PY{p}{,} \PY{n}{qubit}\PY{p}{,} \PY{n}{cbit}\PY{p}{)}\PY{p}{:}
    \PY{l+s+sd}{\PYZsq{}\PYZsq{}\PYZsq{}}
\PY{l+s+sd}{    Measures \PYZsq{}\PYZsq{}qubit in the X\PYZhy{}basis and store the result in \PYZsq{}cbit\PYZsq{}}
\PY{l+s+sd}{    \PYZsq{}\PYZsq{}\PYZsq{}}
    \PY{n}{qc}\PY{o}{.}\PY{n}{h}\PY{p}{(}\PY{n}{qubit}\PY{p}{)}  \PY{c+c1}{\PYZsh{}\PYZsh{} transforms from the Z\PYZhy{}basis to the X\PYZhy{}basis}
    \PY{n}{qc}\PY{o}{.}\PY{n}{measure}\PY{p}{(}\PY{n}{qubit}\PY{p}{,} \PY{n}{cbit}\PY{p}{)}
    \PY{k}{return} \PY{n}{qc}
\end{Verbatim}
\end{tcolorbox}

    \begin{tcolorbox}[breakable, size=fbox, boxrule=1pt, pad at break*=1mm,colback=cellbackground, colframe=cellborder]
\prompt{In}{incolor}{55}{\boxspacing}
\begin{Verbatim}[commandchars=\\\{\}]
\PY{n}{initial\PYZus{}state} \PY{o}{=}\PY{p}{[}\PY{l+m+mi}{1}\PY{o}{/}\PY{n}{sqrt}\PY{p}{(}\PY{l+m+mi}{2}\PY{p}{)}\PY{p}{,} \PY{o}{\PYZhy{}}\PY{l+m+mi}{1}\PY{o}{/}\PY{n}{sqrt}\PY{p}{(}\PY{l+m+mi}{2}\PY{p}{)}\PY{p}{]}
\PY{n}{qc} \PY{o}{=} \PY{n}{QuantumCircuit}\PY{p}{(}\PY{l+m+mi}{1}\PY{p}{,}\PY{l+m+mi}{1}\PY{p}{)} \PY{c+c1}{\PYZsh{}\PYZsh{}notice the addition of cbit for the measurements}
\PY{n}{qc}\PY{o}{.}\PY{n}{initialize}\PY{p}{(}\PY{n}{initial\PYZus{}state}\PY{p}{,} \PY{l+m+mi}{0}\PY{p}{)}
\PY{n}{x\PYZus{}measurement}\PY{p}{(}\PY{n}{qc}\PY{p}{,} \PY{l+m+mi}{0}\PY{p}{,} \PY{l+m+mi}{0}\PY{p}{)} \PY{c+c1}{\PYZsh{}measure qubit 0 to classical bit 0}
\PY{n}{qc}\PY{o}{.}\PY{n}{draw}\PY{p}{(}\PY{p}{)}
\end{Verbatim}
\end{tcolorbox}
 
            
\prompt{Out}{outcolor}{55}{}
    
    \begin{center}
    \adjustimage{max size={0.9\linewidth}{0.9\paperheight}}{output_124_0.pdf}
    \end{center}
    { \hspace*{\fill} \\}
    

    Remember that \(X=HZH\). What this is doing is:

H-gate switches the qubit to X-basis, \[
\begin{eqnarray}
H|0\rangle &= |+\rangle\\
H|1\rangle &= |-\rangle
\end{eqnarray}
\]

Z-gate performs a NOT in the X-basis, \[
\begin{eqnarray}
Z|+\rangle &= |-\rangle\\
Z|-\rangle &= |+\rangle
\end{eqnarray}
\]

and the final H-gate returns the qubit to the Z-basis. \[
\begin{eqnarray}
H|+\rangle &= |0\rangle\\
H|-\rangle &= |1\rangle
\end{eqnarray}
\]

    \textbf{Quick exercises}:

\#1 If we initialize our qubit in the state \(|+\rangle\), what is the
probability of measuring it in state \(|-\rangle\) ?

\#2 Use Qiskit to display the probability of measuring a \(|0\rangle\)
qubit in the states \(|+\rangle\) and \(|-\rangle\) (Hint: you might
want to use .get\_counts() and plot\_histogram()).

\#3 Try to create a function that measures in the Y-basis.

    \begin{tcolorbox}[breakable, size=fbox, boxrule=1pt, pad at break*=1mm,colback=cellbackground, colframe=cellborder]
\prompt{In}{incolor}{56}{\boxspacing}
\begin{Verbatim}[commandchars=\\\{\}]
\PY{c+c1}{\PYZsh{} Probability is zero}
\PY{n}{initial\PYZus{}state} \PY{o}{=}\PY{p}{[}\PY{l+m+mi}{1}\PY{o}{/}\PY{n}{sqrt}\PY{p}{(}\PY{l+m+mi}{2}\PY{p}{)}\PY{p}{,} \PY{l+m+mi}{1}\PY{o}{/}\PY{n}{sqrt}\PY{p}{(}\PY{l+m+mi}{2}\PY{p}{)}\PY{p}{]}
\PY{n}{qc} \PY{o}{=} \PY{n}{QuantumCircuit}\PY{p}{(}\PY{l+m+mi}{1}\PY{p}{,}\PY{l+m+mi}{1}\PY{p}{)} \PY{c+c1}{\PYZsh{}\PYZsh{}notice the addition of cbit for the measurements}
\PY{n}{qc}\PY{o}{.}\PY{n}{initialize}\PY{p}{(}\PY{n}{initial\PYZus{}state}\PY{p}{,} \PY{l+m+mi}{0}\PY{p}{)}
\PY{n}{x\PYZus{}measurement}\PY{p}{(}\PY{n}{qc}\PY{p}{,} \PY{l+m+mi}{0}\PY{p}{,} \PY{l+m+mi}{0}\PY{p}{)} \PY{c+c1}{\PYZsh{}measure qubit 0 to classical bit 0}
\PY{n}{counts} \PY{o}{=} \PY{n}{sim}\PY{o}{.}\PY{n}{run}\PY{p}{(}\PY{n}{qc}\PY{p}{)}\PY{o}{.}\PY{n}{result}\PY{p}{(}\PY{p}{)}\PY{o}{.}\PY{n}{get\PYZus{}counts}\PY{p}{(}\PY{p}{)}
\PY{n}{plot\PYZus{}histogram}\PY{p}{(}\PY{n}{counts}\PY{p}{)}
\PY{c+c1}{\PYZsh{}qc.draw()}
\end{Verbatim}
\end{tcolorbox}
 
            
\prompt{Out}{outcolor}{56}{}
    
    \begin{center}
    \adjustimage{max size={0.9\linewidth}{0.9\paperheight}}{output_127_0.pdf}
    \end{center}
    { \hspace*{\fill} \\}
    

    \begin{tcolorbox}[breakable, size=fbox, boxrule=1pt, pad at break*=1mm,colback=cellbackground, colframe=cellborder]
\prompt{In}{incolor}{57}{\boxspacing}
\begin{Verbatim}[commandchars=\\\{\}]
\PY{n}{qc} \PY{o}{=} \PY{n}{QuantumCircuit}\PY{p}{(}\PY{l+m+mi}{1}\PY{p}{,}\PY{l+m+mi}{1}\PY{p}{)}
\PY{n}{x\PYZus{}measurement}\PY{p}{(}\PY{n}{qc}\PY{p}{,} \PY{l+m+mi}{0}\PY{p}{,} \PY{l+m+mi}{0}\PY{p}{)}
\PY{n}{counts} \PY{o}{=} \PY{n}{sim}\PY{o}{.}\PY{n}{run}\PY{p}{(}\PY{n}{qc}\PY{p}{)}\PY{o}{.}\PY{n}{result}\PY{p}{(}\PY{p}{)}\PY{o}{.}\PY{n}{get\PYZus{}counts}\PY{p}{(}\PY{p}{)}
\PY{n}{plot\PYZus{}histogram}\PY{p}{(}\PY{n}{counts}\PY{p}{)}

\PY{c+c1}{\PYZsh{}\PYZsh{} Probability is 1/2 and 1/2}
\end{Verbatim}
\end{tcolorbox}
 
            
\prompt{Out}{outcolor}{57}{}
    
    \begin{center}
    \adjustimage{max size={0.9\linewidth}{0.9\paperheight}}{output_128_0.pdf}
    \end{center}
    { \hspace*{\fill} \\}
    

    \begin{tcolorbox}[breakable, size=fbox, boxrule=1pt, pad at break*=1mm,colback=cellbackground, colframe=cellborder]
\prompt{In}{incolor}{58}{\boxspacing}
\begin{Verbatim}[commandchars=\\\{\}]
\PY{k}{def} \PY{n+nf}{y\PYZus{}measurement}\PY{p}{(}\PY{n}{qc}\PY{p}{,}\PY{n}{qubit}\PY{p}{,}\PY{n}{cbit}\PY{p}{)}\PY{p}{:}
    \PY{n}{qc}\PY{o}{.}\PY{n}{sdg}\PY{p}{(}\PY{n}{qubit}\PY{p}{)}
    \PY{n}{qc}\PY{o}{.}\PY{n}{h}\PY{p}{(}\PY{n}{qubit}\PY{p}{)}
    \PY{n}{qc}\PY{o}{.}\PY{n}{measure}\PY{p}{(}\PY{n}{qubit}\PY{p}{,}\PY{n}{cbit}\PY{p}{)}
    \PY{k}{return} \PY{n}{qc}

\PY{n}{circuit} \PY{o}{=} \PY{n}{QuantumCircuit}\PY{p}{(}\PY{l+m+mi}{1}\PY{p}{,}\PY{l+m+mi}{1}\PY{p}{)}
\PY{n}{circuit}\PY{o}{.}\PY{n}{h}\PY{p}{(}\PY{l+m+mi}{0}\PY{p}{)}
\PY{n}{circuit}\PY{o}{.}\PY{n}{barrier}\PY{p}{(}\PY{p}{)}


\PY{n}{y\PYZus{}measurement}\PY{p}{(}\PY{n}{circuit}\PY{p}{,} \PY{l+m+mi}{0}\PY{p}{,} \PY{l+m+mi}{0}\PY{p}{)}
\PY{n}{counts} \PY{o}{=} \PY{n}{sim}\PY{o}{.}\PY{n}{run}\PY{p}{(}\PY{n}{qc}\PY{p}{)}\PY{o}{.}\PY{n}{result}\PY{p}{(}\PY{p}{)}\PY{o}{.}\PY{n}{get\PYZus{}counts}\PY{p}{(}\PY{p}{)}
\PY{n}{plot\PYZus{}histogram}\PY{p}{(}\PY{n}{counts}\PY{p}{)}
\end{Verbatim}
\end{tcolorbox}
 
            
\prompt{Out}{outcolor}{58}{}
    
    \begin{center}
    \adjustimage{max size={0.9\linewidth}{0.9\paperheight}}{output_129_0.pdf}
    \end{center}
    { \hspace*{\fill} \\}
    

    \textbf{Whatever state our quantum system is in, there is always a
measurement that has a deterministic outcome.}

    \begin{tcolorbox}[breakable, size=fbox, boxrule=1pt, pad at break*=1mm,colback=cellbackground, colframe=cellborder]
\prompt{In}{incolor}{ }{\boxspacing}
\begin{Verbatim}[commandchars=\\\{\}]

\end{Verbatim}
\end{tcolorbox}

    \hypertarget{the-p-gate}{%
\subsubsection{The P-Gate}\label{the-p-gate}}

    P-gate (phase gate) is parameterised by a rotation around the Z-axis.

\[
P(\phi) = 
\begin{bmatrix}
1 & 0\\
0 & e^{i\phi}
\end{bmatrix}
, \phi \  \mathrm{in} \ \mathbb{R}
\]

    \begin{tcolorbox}[breakable, size=fbox, boxrule=1pt, pad at break*=1mm,colback=cellbackground, colframe=cellborder]
\prompt{In}{incolor}{59}{\boxspacing}
\begin{Verbatim}[commandchars=\\\{\}]
\PY{k+kn}{from} \PY{n+nn}{qiskit\PYZus{}textbook}\PY{n+nn}{.}\PY{n+nn}{widgets} \PY{k+kn}{import} \PY{n}{gate\PYZus{}demo}
\PY{n}{gate\PYZus{}demo}\PY{p}{(}\PY{n}{gates}\PY{o}{=}\PY{l+s+s1}{\PYZsq{}}\PY{l+s+s1}{pauli+h+p}\PY{l+s+s1}{\PYZsq{}}\PY{p}{)}
\end{Verbatim}
\end{tcolorbox}

    
    \begin{Verbatim}[commandchars=\\\{\}]
VBox(children=(HBox(children=(Button(description='X', layout=Layout(height='3em', width='3em'), style=ButtonSt…
    \end{Verbatim}

    
    
    \begin{Verbatim}[commandchars=\\\{\}]
Image(value=b'\textbackslash{}x89PNG\textbackslash{}r\textbackslash{}n\textbackslash{}x1a\textbackslash{}n\textbackslash{}x00\textbackslash{}x00\textbackslash{}x00\textbackslash{}rIHDR\textbackslash{}x00\textbackslash{}x00\textbackslash{}x01 \textbackslash{}x00\textbackslash{}x00\textbackslash{}x01 \textbackslash{}x08\textbackslash{}x06\textbackslash{}x00\textbackslash{}x00\textbackslash{}x00\textbackslash{}x14\textbackslash{}x83\textbackslash{}xae\textbackslash{}x8…
    \end{Verbatim}

    
    It is defined in Qiskit as

    \begin{tcolorbox}[breakable, size=fbox, boxrule=1pt, pad at break*=1mm,colback=cellbackground, colframe=cellborder]
\prompt{In}{incolor}{60}{\boxspacing}
\begin{Verbatim}[commandchars=\\\{\}]
\PY{n}{qc} \PY{o}{=} \PY{n}{QuantumCircuit}\PY{p}{(}\PY{l+m+mi}{1}\PY{p}{)}
\PY{n}{qc}\PY{o}{.}\PY{n}{p}\PY{p}{(}\PY{n}{pi}\PY{o}{/}\PY{l+m+mi}{4}\PY{p}{,} \PY{l+m+mi}{0}\PY{p}{)}
\PY{n}{qc}\PY{o}{.}\PY{n}{draw}\PY{p}{(}\PY{p}{)}
\end{Verbatim}
\end{tcolorbox}
 
            
\prompt{Out}{outcolor}{60}{}
    
    \begin{center}
    \adjustimage{max size={0.9\linewidth}{0.9\paperheight}}{output_136_0.pdf}
    \end{center}
    { \hspace*{\fill} \\}
    

    Notice the Z-gate is a special case of the P-gate, for \(\phi=\pi\).
There are also 3 other important special cases: the I, S, and T-Gates

    \begin{tcolorbox}[breakable, size=fbox, boxrule=1pt, pad at break*=1mm,colback=cellbackground, colframe=cellborder]
\prompt{In}{incolor}{ }{\boxspacing}
\begin{Verbatim}[commandchars=\\\{\}]

\end{Verbatim}
\end{tcolorbox}

    \hypertarget{the-i-s-and-t-gates}{%
\subsubsection{The I, S, and T-Gates}\label{the-i-s-and-t-gates}}

    \hypertarget{the-i-gate}{%
\paragraph{The I-Gate}\label{the-i-gate}}

    Identity gate

\[
I = 
\begin{bmatrix}
1 & 0\\
0 & 1
\end{bmatrix}
\]

    The reaspm to consider this a gate (which does not do anything on a
qubit) is its use in calculations, such as \(I=XX\), etc.

\textbf{Quick exercise}: Which are the eigenstates of the I-gate?

All statevectors should be eigenstates of the I-gate, with eigenvalue
\(=1\).

    \hypertarget{the-s-gate}{%
\paragraph{The S-Gate}\label{the-s-gate}}

    \(S\)-gate, also known as \(\sqrt{Z}\)-gate \(\rightarrow \phi=\pi/2\).
Importantly, \textbf{it is not its own inverse}, i.e., \[SS \neq I\].

\(S^{\dagger}\)-gate, also known as \(\sqrt{Z}^{\dagger}\)-gate
\(\rightarrow \phi=-\pi/2\)

\[
SS|q\rangle = Z |q\rangle
\] the reason for the \(\sqrt{Z}\)-gate.

    \begin{tcolorbox}[breakable, size=fbox, boxrule=1pt, pad at break*=1mm,colback=cellbackground, colframe=cellborder]
\prompt{In}{incolor}{61}{\boxspacing}
\begin{Verbatim}[commandchars=\\\{\}]
\PY{n}{qc} \PY{o}{=} \PY{n}{QuantumCircuit}\PY{p}{(}\PY{l+m+mi}{1}\PY{p}{)}
\PY{n}{qc}\PY{o}{.}\PY{n}{s}\PY{p}{(}\PY{l+m+mi}{0}\PY{p}{)}
\PY{n}{qc}\PY{o}{.}\PY{n}{sdg}\PY{p}{(}\PY{l+m+mi}{0}\PY{p}{)}
\PY{n}{qc}\PY{o}{.}\PY{n}{draw}\PY{p}{(}\PY{p}{)}
\end{Verbatim}
\end{tcolorbox}
 
            
\prompt{Out}{outcolor}{61}{}
    
    \begin{center}
    \adjustimage{max size={0.9\linewidth}{0.9\paperheight}}{output_145_0.pdf}
    \end{center}
    { \hspace*{\fill} \\}
    

    \hypertarget{the-t-gate}{%
\paragraph{The T-Gate}\label{the-t-gate}}

    \(T\)-gate, sometimes known as \(\sqrt[4]{Z}\)-gate,
\(\rightarrow \phi=\pi/4\)

\[
T = 
\begin{bmatrix}
1 & 0\\
0 & \exp(i\pi/4)
\end{bmatrix},
\ \ \ 
T\dagger = 
\begin{bmatrix}
1 & 0\\
0 & \exp(-i\pi/4)
\end{bmatrix}
\]

    \begin{tcolorbox}[breakable, size=fbox, boxrule=1pt, pad at break*=1mm,colback=cellbackground, colframe=cellborder]
\prompt{In}{incolor}{62}{\boxspacing}
\begin{Verbatim}[commandchars=\\\{\}]
\PY{n}{qc} \PY{o}{=} \PY{n}{QuantumCircuit}\PY{p}{(}\PY{l+m+mi}{1}\PY{p}{)}
\PY{n}{qc}\PY{o}{.}\PY{n}{t}\PY{p}{(}\PY{l+m+mi}{0}\PY{p}{)}
\PY{n}{qc}\PY{o}{.}\PY{n}{tdg}\PY{p}{(}\PY{l+m+mi}{0}\PY{p}{)}
\PY{n}{qc}\PY{o}{.}\PY{n}{draw}\PY{p}{(}\PY{p}{)}
\end{Verbatim}
\end{tcolorbox}
 
            
\prompt{Out}{outcolor}{62}{}
    
    \begin{center}
    \adjustimage{max size={0.9\linewidth}{0.9\paperheight}}{output_148_0.pdf}
    \end{center}
    { \hspace*{\fill} \\}
    

    \begin{tcolorbox}[breakable, size=fbox, boxrule=1pt, pad at break*=1mm,colback=cellbackground, colframe=cellborder]
\prompt{In}{incolor}{63}{\boxspacing}
\begin{Verbatim}[commandchars=\\\{\}]
\PY{k+kn}{from} \PY{n+nn}{qiskit\PYZus{}textbook}\PY{n+nn}{.}\PY{n+nn}{widgets} \PY{k+kn}{import} \PY{n}{gate\PYZus{}demo}
\PY{n}{gate\PYZus{}demo}\PY{p}{(}\PY{p}{)}
\end{Verbatim}
\end{tcolorbox}

    
    \begin{Verbatim}[commandchars=\\\{\}]
VBox(children=(HBox(children=(Button(description='I', layout=Layout(height='3em', width='3em'), style=ButtonSt…
    \end{Verbatim}

    
    
    \begin{Verbatim}[commandchars=\\\{\}]
Image(value=b'\textbackslash{}x89PNG\textbackslash{}r\textbackslash{}n\textbackslash{}x1a\textbackslash{}n\textbackslash{}x00\textbackslash{}x00\textbackslash{}x00\textbackslash{}rIHDR\textbackslash{}x00\textbackslash{}x00\textbackslash{}x01 \textbackslash{}x00\textbackslash{}x00\textbackslash{}x01 \textbackslash{}x08\textbackslash{}x06\textbackslash{}x00\textbackslash{}x00\textbackslash{}x00\textbackslash{}x14\textbackslash{}x83\textbackslash{}xae\textbackslash{}x8…
    \end{Verbatim}

    
    \hypertarget{the-general-u-gate}{%
\subsubsection{The General U-Gate}\label{the-general-u-gate}}

    \(U\)-gate, most general of the single qubit gates can be parameterised
by 3-parameters:

\[
U(\theta, \phi, \lambda) =
\begin{bmatrix}
\cos(\theta/2) & e^{-i\lambda}\sin(\theta/2)\\
e^{i\phi}\sin(\theta/2) & e^{i(\phi+\lambda)}\cos(\theta/2)
\end{bmatrix}
\]

where the specific cases become

\[
U(\frac{\pi}{2}, 0, \pi) = \frac{1}{\sqrt{2}}
\begin{bmatrix}
1 & 1 \\
1 & -1
\end{bmatrix}
= H,
\ \ \ \
U(0, 0, \lambda) = 
\begin{bmatrix}
1 & 0 \\
0 & e^{i\lambda}
\end{bmatrix}
= P
\ \ \ \
\]

    \begin{tcolorbox}[breakable, size=fbox, boxrule=1pt, pad at break*=1mm,colback=cellbackground, colframe=cellborder]
\prompt{In}{incolor}{64}{\boxspacing}
\begin{Verbatim}[commandchars=\\\{\}]
\PY{n}{qc} \PY{o}{=} \PY{n}{QuantumCircuit}\PY{p}{(}\PY{l+m+mi}{1}\PY{p}{)}
\PY{n}{qc}\PY{o}{.}\PY{n}{u}\PY{p}{(}\PY{n}{pi}\PY{o}{/}\PY{l+m+mi}{2}\PY{p}{,} \PY{l+m+mi}{0}\PY{p}{,} \PY{n}{pi}\PY{p}{,} \PY{l+m+mi}{0}\PY{p}{)}\PY{c+c1}{\PYZsh{}\PYZsh{} 3 parameters and qubit to be applied to}
\PY{n}{qc}\PY{o}{.}\PY{n}{draw}\PY{p}{(}\PY{p}{)}
\end{Verbatim}
\end{tcolorbox}
 
            
\prompt{Out}{outcolor}{64}{}
    
    \begin{center}
    \adjustimage{max size={0.9\linewidth}{0.9\paperheight}}{output_152_0.pdf}
    \end{center}
    { \hspace*{\fill} \\}
    

    \begin{tcolorbox}[breakable, size=fbox, boxrule=1pt, pad at break*=1mm,colback=cellbackground, colframe=cellborder]
\prompt{In}{incolor}{65}{\boxspacing}
\begin{Verbatim}[commandchars=\\\{\}]
\PY{n}{qc}\PY{o}{.}\PY{n}{save\PYZus{}statevector}\PY{p}{(}\PY{p}{)}
\PY{n}{state} \PY{o}{=} \PY{n}{sim}\PY{o}{.}\PY{n}{run}\PY{p}{(}\PY{n}{qc}\PY{p}{)}\PY{o}{.}\PY{n}{result}\PY{p}{(}\PY{p}{)}\PY{o}{.}\PY{n}{get\PYZus{}statevector}\PY{p}{(}\PY{p}{)}
\PY{n}{plot\PYZus{}bloch\PYZus{}multivector}\PY{p}{(}\PY{n}{state}\PY{p}{)}
\end{Verbatim}
\end{tcolorbox}
 
            
\prompt{Out}{outcolor}{65}{}
    
    \begin{center}
    \adjustimage{max size={0.9\linewidth}{0.9\paperheight}}{output_153_0.pdf}
    \end{center}
    { \hspace*{\fill} \\}
    

    Qiskit also provides the X equivalent of S gates, SX-gate and SXdg-gate,
which do a quarter turn around the X-axis.

\textbf{Before running on real IBM quantum hardware, all single qubit
ops are compiled down to \(I,X, SX\), and \(R_z\), whihc are called
\emph{physical gates}.}

    \hypertarget{the-case-for-quantum}{%
\subsection{The Case for Quantum}\label{the-case-for-quantum}}

    \hypertarget{complexity-of-adding}{%
\subsubsection{Complexity of adding}\label{complexity-of-adding}}

\[n \leq c(n) \leq 2n \longrightarrow c(n) = O(n)\]

\#\#\# Big O notation

For functions \(f(x)\) and \(g(x)\) and parameter \(x\), the statement
\(f(x) = O(g(x))\) means that \(\exists\) some finite \(M>0\) and
\(x_0\) such that

\[
 f(x) \leq Mg(x), \forall x > x_0
 \]

    Sclaing as functions of the input size \(n\)

    \hypertarget{complexity-theory}{%
\subsubsection{Complexity Theory}\label{complexity-theory}}

Multiplication: \(O(n^2)\)

Factorization: \(O(e^{n^{1/3}})\)

Searching database:\(O(n)\)

    Formally, defining the complexity of an algorithm depends on the exact
theoretical model for computation we are using. Each model has a set of
basic operations, known as primitive operations, with which any
algorithm can be expressed.

For \textbf{Boolean circuits}, as we considered in the first section,
the primitive operations are the logic gates.

For \textbf{Turing machines}, a hypothetical form of computer proposed
by Alan Turing, we imagine a device stepping through and manipulating
information stored on a tape.

The \textbf{RAM model} has a more complex set of primitive operations
and acts as an idealized form of the computers we use every day.

All these are models of digital computation, based on discretized
manipulations of discrete values. Different as they may seem from each
other, \emph{it turns out that it is very easy for each of them to
simulate the others}.

\hypertarget{beyond-digital-computers}{%
\subsubsection{Beyond digital
computers}\label{beyond-digital-computers}}

\textbf{Digital computers}: discrete values (0-1, e.g.). Can detect and
correct errors relatively easy.

\textbf{Analog computers}: precise manipulation of continuoulsy varying
parameters. Issue: arbitrary precision.

Ideally: robusteness of digital with subtle manipulations of analog.

Quantum computing is the only known technology exponentially fatster
than classical computers for certain tasks.

\hypertarget{when-to-use-qc}{%
\subsubsection{When to use QC}\label{when-to-use-qc}}

Novel algorithms: One way in which this can be done is when we have some
function for which we want to determine a global property. For example,
if we want to find the value of some parameter \(x\) for which some
function \(f(x)\) is a minimum, or the period of the function if
\(f(x)\) is periodic.

Superposition of states to induce quantum interference and reveal global
property.

Ex.

\textbf{Grover's search algorithm}: \(O(n)\) to \(O(n^{1/2})\).

\textbf{Shor's factorization algorithm}: \(O(e^{n^{1/3}})\) to
\(O(n^3)\)

\textbf{Solve quantum problems}: dimensional need.

Particularly promising are those problems for which classical algorithms
face inherent scaling limits and which do not require a large classical
dataset to be loaded.

See
https://www.cs.virginia.edu/\textasciitilde robins/The\_Limits\_of\_Quantum\_Computers.pdf

    \begin{tcolorbox}[breakable, size=fbox, boxrule=1pt, pad at break*=1mm,colback=cellbackground, colframe=cellborder]
\prompt{In}{incolor}{ }{\boxspacing}
\begin{Verbatim}[commandchars=\\\{\}]

\end{Verbatim}
\end{tcolorbox}

    \begin{tcolorbox}[breakable, size=fbox, boxrule=1pt, pad at break*=1mm,colback=cellbackground, colframe=cellborder]
\prompt{In}{incolor}{ }{\boxspacing}
\begin{Verbatim}[commandchars=\\\{\}]

\end{Verbatim}
\end{tcolorbox}

    \begin{tcolorbox}[breakable, size=fbox, boxrule=1pt, pad at break*=1mm,colback=cellbackground, colframe=cellborder]
\prompt{In}{incolor}{ }{\boxspacing}
\begin{Verbatim}[commandchars=\\\{\}]

\end{Verbatim}
\end{tcolorbox}

    \begin{tcolorbox}[breakable, size=fbox, boxrule=1pt, pad at break*=1mm,colback=cellbackground, colframe=cellborder]
\prompt{In}{incolor}{66}{\boxspacing}
\begin{Verbatim}[commandchars=\\\{\}]
\PY{k+kn}{import} \PY{n+nn}{qiskit}\PY{n+nn}{.}\PY{n+nn}{tools}\PY{n+nn}{.}\PY{n+nn}{jupyter}
\PY{o}{\PYZpc{}}\PY{k}{qiskit\PYZus{}version\PYZus{}table}
\end{Verbatim}
\end{tcolorbox}

    \begin{Verbatim}[commandchars=\\\{\}]
/Users/ufranca/opt/anaconda3/lib/python3.8/site-
packages/qiskit/aqua/\_\_init\_\_.py:86: DeprecationWarning: The package qiskit.aqua
is deprecated. It was moved/refactored to qiskit-terra For more information see
<https://github.com/Qiskit/qiskit-aqua/blob/main/README.md\#migration-guide>
  warn\_package('aqua', 'qiskit-terra')
    \end{Verbatim}

    
    \begin{Verbatim}[commandchars=\\\{\}]
<IPython.core.display.HTML object>
    \end{Verbatim}

    
    \begin{tcolorbox}[breakable, size=fbox, boxrule=1pt, pad at break*=1mm,colback=cellbackground, colframe=cellborder]
\prompt{In}{incolor}{ }{\boxspacing}
\begin{Verbatim}[commandchars=\\\{\}]

\end{Verbatim}
\end{tcolorbox}


    % Add a bibliography block to the postdoc
    
    
    
\end{document}
